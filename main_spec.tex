\documentclass[10pt]{book}

\input{utils/packages}

\definecolor{ListingBG}{rgb}{0.91,0.91,0.91}
\definecolor{shadecolor}{rgb}{0.92,0.92,0.92}

\hyphenation{Open-SHMEM}

\renewcommand{\chaptername}{Chapter}
\renewcommand{\appendixname}{Annex}

% Place some penalty for doing the break
% The penalty for a ``\gb'' should be greater than a \hyphenpenalty.
% \hyphenpenalty is 50 in plain.tex.
\def\gb{\penalty10000\hskip 0pt plus 8em\penalty4800\hskip 0pt plus-8em%
\penalty10000}

% This macro enables that all "_" (underscore) characters in the pfd
% file are searchable, and that cut&paste will copy the "_" as underscore.
% Without the following macro, the \_ is treated in searches and cut&paste
% as a " " (space character).
% This macro does not modify the behavior of _ in math or in verbatim
% environments. In verbatim environments, the "_" is always treated
% as a searchable character.
%
\DeclareRobustCommand{\_}{\texttt{\char`\_}}
%

\def\colorswapnt{\colorlet{saved}{.}\color{ForestGreen}}
\def\colorswapot{\colorlet{saved}{.}\color{red}}
\def\prevcolor{\color{saved}}

\newcommand{\newtext}[1]{\textcolor{ForestGreen}{#1}}
\newcommand{\oldtext}[1]{\textcolor{magenta}{\sout{#1}}}
\newcommand{\insertDocVersion}{1.5}
\newcommand{\OSH}{\emph{OpenSHMEM}}
\newcommand{\openshmem}{{Open\-SHMEM}\xspace}
\newcommand{\FUNC}[1]{\textit{#1}}
\newcommand{\VAR}[1]{\textit{#1}}
\newcommand{\CONST}[1]{\textit{#1}}
\newcommand{\const}[1]{\protect\gb\protect{\textsf{\small #1}}\index{CONST:#1}}
\newcommand{\CorCpp}{\textit{C/C++}\xspace}
\newcommand{\CorCppFor}{\textit{C/C++/Fortran}\xspace}
\newcommand{\Fortran}{\textit{Fortran}}
\newcommand{\Clang}{\textit{C}}
\newcommand{\Cpp}{\textit{C++}}
\newcommand{\Celev}{\textit{C11}}
\newcommand{\TYPE}{\emph{TYPE}}
\newcommand{\TYPENAME}{\emph{TYPENAME}}
\newcommand{\SIZE}{\emph{SIZE}}

\newcommand{\source}{\textit{source}}
\newcommand{\target}{\textit{target}}
\newcommand{\PUT}{\textit{Put}}
\newcommand{\GET}{\textit{Get}}
\newcommand{\OPR}[1]{\textit{#1}}
\newcommand{\dest}{\textit{dest}}
\newcommand{\barrier}{\FUNC{SHMEM\_BARRIER}\xspace} % why here an not others?
\newcommand{\barrierall}{\FUNC{SHMEM\_BARRIER\_ALL}\xspace} % why here an not others?
\newcommand{\broadcast}{\FUNC{SHMEM\_BROADCAST}}
\newcommand{\collect}{\FUNC{SHMEM\_COLLECT}}
\newcommand{\fcollect}{\FUNC{SHMEM\_FCOLLECT}}
\newcommand{\reduction}{\textit{Reduction Operations}}
\newcommand{\alltoall}{\FUNC{SHMEM\_ALLTOALL}}
\newcommand{\alltoalls}{\FUNC{SHMEM\_ALLTOALLS}}
\newcommand{\activeset}{\textit{Active~set}\xspace} % why here and not others?
\newcommand{\shmemprefix}{\textit{SHMEM\_}}
\newcommand{\shmemprefixLC}{\textit{shmem\_}}
\newcommand{\shmemprefixC}{\textit{\_SHMEM\_}}
\newcommand{\ith}{${\textit{i}^{\text{\tiny th}}}$}
\newcommand{\jth}{${\textit{j}^{\text{\tiny th}}}$}
\newcommand{\kth}{${\textit{k}^{\text{\tiny th}}}$}
\newcommand{\lth}{${\textit{l}^{\text{\tiny th}}}$}

\begin{acronym}
\acro{RMA}{\emph{Remote Memory Access}}
\acro{RMO}{\emph{Remote Memory Operation}}
\acro{AMO}{\emph{Atomic Memory Operation}}
\acro{PE}{\emph{Processing Element}}
\acrodefplural{PE}[PEs]{\emph{Processing Elements}}
\acro{PGAS}{\emph{Partitioned Global Address Space}}
\acro{API}{\emph{Application Programming Interface}}
\acro{MPI}{\emph{Message Passing Interface}}
\acro{SPMD}{\emph{Single Program Multiple Data}}
\acro{UH}{University of Houston}
\acro{UO}{University of Oregon}
\acro{ORNL}{Oak Ridge National Laboratory}
\acro{LANL}{Los Alamos National Laboratory}
\acro{ESSC}{Extreme Scale Systems Center}
\acro{OSSS}{Open Software System Solutions}
\acro{DoD}{U.S. Department of Defense}
\end{acronym}


%
% This is used to put line numbers on plain pages.  Used in draft.tex
%
\makeatletter

\def\withlinenumbers{\relax
  \def\@evenfoot{\hbox to 0pt{\hss\LineNumberRuler\hskip 1.5pc}\hfil}\relax
  \def\@oddfoot{\hfil\hbox to 0pt{\hskip 1.5pc\LineNumberRuler\hss}}}

\def\LineNumberRuler{\vbox to 0pt{\vss\normalsize \baselineskip13.6pt
    \lineskip 1pt \normallineskip 1pt \def\baselinestretch{1}\relax
    \LNR{1}\LNR{2}\LNR{3}\LNR{4}\LNR{5}\LNR{6}\LNR{7}\LNR{8}\LNR{9}
    \LNR{10}\LNR{11}\LNR{12}\LNR{13}\LNR{14}
        \LNR{15}\LNR{16}\LNR{17}\LNR{18}\LNR{19}
    \LNR{20}\LNR{21}\LNR{22}\LNR{23}\LNR{24}
        \LNR{25}\LNR{26}\LNR{27}\LNR{28}\LNR{29}
    \LNR{30}\LNR{31}\LNR{32}\LNR{33}\LNR{34}\LNR{35}
        \LNR{36}\LNR{37}\LNR{38}\LNR{39}
    \LNR{40}\LNR{41}\LNR{42}\LNR{43}\LNR{44}
        \LNR{45}\LNR{46}\LNR{47}\LNR{48}
    \vskip 31pt}}
\def\LNR#1{\hbox to 1pc{\hfil\tiny#1\hfil}}

\def\ps@plainwithlinenumbers{\let\@mkboth\@gobbletwo
     \def\@oddhead{}
     \def\@oddfoot{\hfil\rm\thepage\hfil
       \hbox to 0pt{\hskip 1.5pc\LineNumberRuler\hss}}
     \def\@evenhead{}
     \def\@evenfoot{\hbox to 0pt{\hss
     \LineNumberRuler\hskip 1.5pc}\rm\hfil\thepage\hfil}}

    % Contents is done with \chapter*{Contents}, so we need to turn off the
    % line numbers in this case.  Easiest to look at def

\newwrite\chappages
\immediate\openout\chappages=chappage.txt
\def\writespace{ }

\def\incontents{0}
\newif\ifcontents
\contentsfalse
\def\chapter{\clearpage \ifcontents\else\thispagestyle{plainwithlinenumbers}\fi
        \write\chappages{Chapter \thechapter\writespace - \the\count0}
        \global\@topnum\z@ \@afterindentfalse \secdef\@chapter\@schapter}

\makeatother

%
% End this is used to put line numbers on plain pages.  Used in draft.tex
%

%
% Use Sans Serif font for sections, etc.
%
\makeatletter
\def\section{\@startsection {section}{1}{\z@}{-3.5ex plus -1ex minus
-.2ex}{2.3ex plus .2ex}{\Large\sf}}
\def\subsection{\@startsection{subsection}{2}{\z@}{-3.25ex plus -1ex minus
-.2ex}{1.5ex plus .2ex}{\large\sf}}
\def\subsubsection{\@startsection{subsubsection}{3}{\z@}{-3.25ex plus
-1ex minus -.2ex}{1.5ex plus .2ex}{\normalsize\sf\bf}}
\def\paragraph{\@startsection {paragraph}{4}{\z@}{3.25ex plus 1ex
minus .2ex}{-1em}{\normalsize\sf}}
\makeatother
%
% End use Sans Serif font for sections, etc.  S. Otto
%


%
% This section is for example code listings
%
\definecolor{gray}{rgb}{0.92,0.92,0.92}

\lstset{ % set defaults for languages not otherwise defined
  breakatwhitespace=false,         % sets if automatic breaks should only happen at whitespace
  basicstyle=\ttfamily\footnotesize,
  breaklines=true,                 % sets automatic line breaking
  escapeinside={|}{|},          % if you want to add LaTeX within your code
  extendedchars=true,              % lets you use non-ASCII characters; for 8-bits
                                   % encodings only, does not work with UTF-8
  keepspaces=true,                 % keeps spaces in text, useful for keeping indentation of code
                                   % (possibly needs columns=flexible)
  morekeywords={*,...},            % if you want to add more keywords to the set
  showspaces=false,                % show spaces everywhere adding particular underscores;
                                   % it overrides 'showstringspaces'
  showstringspaces=false,          % underline spaces within strings only
  showtabs=false,                  % show tabs within strings adding particular underscores
}

\def\StandardListing {
  \lstset {
    breakatwhitespace=false,         % sets if automatic breaks should only happen at whitespace
    basicstyle=\ttfamily\footnotesize,
    breaklines=true,                 % sets automatic line breaking
    escapeinside={\%*}{*)},          % if you want to add LaTeX within your code
    extendedchars=true,              % lets you use non-ASCII characters; for 8-bits
                                     % encodings only, does not work with UTF-8
    keepspaces=true,                 % keeps spaces in text, useful for keeping
                                     % indentation of code (possibly needs columns=flexible)
    morekeywords={*,...},            % if you want to add more keywords to the set
    showspaces=false,                % show spaces everywhere adding particular underscores;
                                     % it overrides 'showstringspaces'
    showstringspaces=false,          % underline spaces within strings only
    showtabs=false,                  % show tabs within strings adding particular underscores
    backgroundcolor=\color{gray},
  }
}

\def\ProgramNumberedListing {
  \StandardListing
  \lstset {
    numbers=left,
    numberstyle=\footnotesize
  }
}

\newcommand{\numberedlisting}[2] {
  \ProgramNumberedListing
  \lstinputlisting[#1]{#2}
  \StandardListing
}

\newcommand{\outputlisting}[2] {
\begin{minipage}{\linewidth}
\vspace{0.1in}
  \lstinputlisting[#1]{#2}
  \StandardListing
\vspace{0.1in}
\end{minipage}
}

\lstdefinelanguage{OSH+C}[]{C}{
  classoffset=1,
  morekeywords={
    SHMEM_BCAST_SYNC_SIZE, SHMEM_SYNC_VALUE,
    start_pes,
    my_pe, _my_pe, shmem_my_pe,
    num_pes, _num_pes, shmem_n_pes,
    shmem_int_p, shmem_short_p, shmem_long_p,
    shmem_int_put, shmem_short_put, shmem_long_put,
    shmem_barrier_all, shmem_barrier,
    shmalloc,  shfree, shrealloc,
    shmem_broadcast32, shmem_broadcast64,
    shmem_short_inc, shmem_int_inc, shmem_long_inc,
    shmem_short_add, shmem_int_add, shmem_long_add,
    shmem_short_finc, shmem_int_finc, shmem_long_finc,
    shmem_short_fadd, shmem_int_fadd, shmem_long_fadd,
    shmem_set_lock, shmem_test_lock, shmem_clear_lock,
    shmem_long_sum_to_all,
    shmem_complexd_sum_to_all,
  },
  keywordstyle=\color{black}\textbf,
  classoffset=0,
  sensitive=true
}

\lstdefinelanguage{OSH2+C}[]{OSH+C}{
  classoffset=1,
  morekeywords={
    shmem_init,
    shmem_finalize,
    shmem_malloc,
    shmem_my_pe,
    shmem_error,
    shmem_global_exit,
  },
  keywordstyle=\color{black}\textbf,
  classoffset=0,
  sensitive=true
}

\lstdefinelanguage{OSH+F}[]{Fortran}{
  classoffset=1,
  morekeywords={
    SHMEM_BCAST_SYNC_SIZE, SHMEM_SYNC_VALUE,
    start_pes,
    my_pe, shmem_my_pe,
    num_pes, shmem_n_pes,
    shmem_int_p, shmem_short_p, shmem_long_p,
    shmem_int_put, shmem_short_put, shmem_long_put,
    shmem_barrier_all, shmem_barrier,
    shpalloc,  shpdeallc, shpclmove,
    shmem_broadcast32, shmem_broadcast64,
    shmem_broadcast4, shmem_broadcast8,
    shmem_short_inc, shmem_int_inc, shmem_long_inc,
    shmem_short_add, shmem_int_add, shmem_long_add,
    shmem_short_finc, shmem_int_finc, shmem_long_finc,
    shmem_short_fadd, shmem_int_fadd, shmem_long_fadd,
    shmem_set_lock, shmem_test_lock, shmem_clear_lock,
    shmem_long_sum_to_all,
  },
  keywordstyle=\color{black}\textbf,
  classoffset=0,
  sensitive=false
}

\lstdefinelanguage{OSH2+F}[]{OSH+F}{
  classoffset=1,
  morekeywords={
    shmem_init,
    shmem_finalize,
    shmem_malloc,
    shmem_my_pe,
    shmem_error,
    shmem_global_exit,
  },
  keywordstyle=\color{black}\textbf,
  classoffset=0,
  sensitive=true
}

%
% End this section is for example code listings
%

%
% Library API description template commands
%

\newcommand{\apisummary}[1]{
    #1
\hfill
}

\newenvironment{apidefinition}{
\begin{description}
\item[SYNOPSIS] \hfill \\ \\
\vspace{-2em}
}
{
\end{description}
}

\lstnewenvironment{Cpp11synopsis}
{
  \textbf{C++11:}
  \lstset{language={C++}, backgroundcolor=\color{gray}, lineskip=2pt,
  morekeywords={size_t, TYPE, noreturn}, aboveskip=0pt, belowskip=0pt}}{}

\lstnewenvironment{C11synopsis}
{
  \textbf{C11:}
  \lstset{language={C}, backgroundcolor=\color{gray}, lineskip=2pt,
  morekeywords={size_t, TYPE, _Noreturn}, aboveskip=0pt, belowskip=0pt}}{}

\lstnewenvironment{CsynopsisCol}
{
  \lstset{language={C}, backgroundcolor=\color{gray}, lineskip=2pt,
  morekeywords={size_t, TYPE, TYPENAME, SIZE}, aboveskip=0pt, belowskip=0pt}}{}


\lstnewenvironment{Csynopsis}
{
  \textbf{C/C++:}
  \lstset{language={C}, backgroundcolor=\color{gray}, lineskip=2pt,
  morekeywords={size_t, TYPE, TYPENAME, SIZE}, aboveskip=0pt, belowskip=0pt}}{}

\lstnewenvironment{CsynopsisST}
{
  \textbf{C/C++:}
  \color{red}
  {\lstset{language={C}, backgroundcolor=\color{gray}, lineskip=2pt,
  morekeywords={size_t}, aboveskip=0pt, belowskip=0pt}}
  }
  {}

\lstnewenvironment{Fsynopsis}
{ \textbf{FORTRAN:}
  \lstset{language={Fortran}, backgroundcolor=\color{gray}, lineskip=3pt,
  deletekeywords=[2]{STATUS},
  deletekeywords=[3]{LOG}, aboveskip=0pt,
  belowskip=0pt}}{}


\newenvironment{envvardefinition}{
\begin{description}
\item[SYNOPSIS] \hfill \\ \\
}
{
\end{description}
}

\lstnewenvironment{envvarname}
{ \textbf{Variable Name:}
  \lstset{backgroundcolor=\color{gray}, lineskip=2pt,
  morekeywords={ID, SIZE, KIND, POLICY, PGSIZE}, aboveskip=0pt, belowskip=0pt}}{}

\newenvironment{envvararguments}{
\newcommand{\envvarargument}[3]{
\begin{tabular}{p{2cm} p{2cm} p{10cm}}
\textbf{##1} & \textit{##2} & {##3} \\
\end{tabular}
}
\hfill

\begin{description}
\item[Memory Traits] \hfill \\
}
{
\hfill
\end{description}
}

\newcommand{\envvarnotes}[1]{
\item[Notes] \hfill \\
    #1
\hfill \\
}

\newenvironment{apiarguments}{
\newcommand{\apiargument}[3]{
\begin{tabular}{p{2cm} p{2cm} p{10cm}}
\textbf{##1} & \textit{##2} & {##3} \\
\end{tabular}
}
\hfill
\item[DESCRIPTION] \hfill

\begin{description}
\item[Arguments] \hfill \\
}
{
\hfill
\end{description}
}

\newcommand{\apidescription}[1]{
\begin{description}
\vspace{-1em}
\item[API description] \hfill \\
    #1
\hfill
}

\newcommand{\apidesctable}[4] {\hfill \\ #1 \\ \\
    \begin{tabular}{p{5cm} p{9cm}}
       \hline
       #2 & #3 \\
       \hline \tabularnewline
       \end{tabular}\\
        #4
}

\newcommand{\apireturnvalues}[1]{
\hfill
\item[Return Values] \hfill \\
    #1
\\
\hfill
}

\newcommand{\apitablerow}[2]{
 \begin{tabular}{p{5cm} p{9cm}}
 #1 & #2 \tabularnewline
  \end{tabular}\\
}

\newcommand{\apinotes}[1]{
\item[Notes] \hfill \\
    #1
\hfill \\
\end{description}
}

\newcommand{\apiimpnotes}[1]{
\begin{description}
\item[Note to implementors] \hfill \\
    #1
\hfill \\
\end{description}
}

\newenvironment{apiexamples}{
\newcommand{\apicexample}[3]{
    ##1
    \lstinputlisting[language={C}, tabsize=2,
    basicstyle=\ttfamily\footnotesize, morekeywords={size_t}]{##2}
    ##3 }
\newcommand{\apifexample}[3]{
    ##1
    \lstinputlisting[language={Fortran}, tabsize=2,
    basicstyle=\ttfamily\footnotesize, deletekeywords={TARGET}]{##2}
    ##3 }
\vspace{-2pt}
\item[EXAMPLES] \hfill \\
\vspace{-2pt}
}
{
}

%
% End library API description template commands
%


\begin{document}

\hypersetup{pageanchor=true,citecolor=blue}

% Set header/footer for opening content
\pagestyle{fancy}
\fancyhead{}
\fancyhead[LE,LO]{\insertDocVersion}
\fancyhead[CO,CE]{--- DRAFT ---}
\SetWatermarkText{DRAFT}
\SetWatermarkScale{1}
\fancyfoot[CE,CO]{\thepage} %affects page numbering for the first pages,
                            %except the first ToC page

\pagenumbering{roman} %sets coverpage and toc page numbers to roman numerals

\thispagestyle{empty}
\begin{center}
\textbf{\LARGE Language Binding Extensions For}\\
\par
\end{center}

\begin{center}
\textbf{\Huge \openshmem}
\par
\end{center}

\begin{center}
\textbf{\LARGE Application Programming Interface}\\
\includegraphics[scale=0.65]{figures/OpenSHMEM_Pound}\\
\url{http://www.openshmem.org/}
\par
\end{center}

\begin{center}
Version \insertDocVersion
\par
\end{center}

\vspace{0.5in}
\begin{center}
\today
\end{center}

\vspace{0.5in}

\vfill{}

\section*{Development by}
\begin{itemize}
\item For a current list of contributors and collaborators please see\\
  \url{http://www.openshmem.org/site/Contributors/}
\item For a current list of OpenSHMEM implementations and tools, please see\\
  \url{http://openshmem.org/site/Links#impl/}

\end{itemize}

\pagebreak{}

\section*{Sponsored by}
\begin{itemize}
\item \ac{DoD}\\
  \url{http://www.defense.gov/ }
\item \ac{ORNL}\\
  \url{http://www.ornl.gov/}
\item \ac{LANL}\\
  \url{http://www.lanl.gov/}
\end{itemize}

\section*{Current Authors and Collaborators}
\begin{itemize}
\item Matthew Baker, \ac{ORNL}
\item Swen Boehm, \ac{ORNL}
\item Aurelien Bouteiller, \ac{UTK}
\item Barbara Chapman, \ac{SBU}
\item Robert Cernohous, Cray Inc.
\item James Culhane, \ac{LANL}
\item Tony Curtis, \ac{SBU}
\item James Dinan, Intel
\item Mike Dubman, Mellanox
\item Karl Feind, \ac{HPE}
\item Manjunath Gorentla Venkata, \ac{ORNL}
\item Max Grossman, Rice University
\item Khaled Hamidouche, \ac{AMD}
\item Jeff Hammond, Intel
\item Yossi Itigin, Mellanox
\item Bryant Lam, \ac{DoD}
\item David Knaak, Cray Inc.
\item Jeff Kuehn, \ac{LANL}
\item Jens Manser, \ac{DoD}
\item Tiffany M. Mintz, \ac{ORNL}
\item David Ozog, Intel
\item Nicholas Park, \ac{DoD}
\item Steve Poole, \ac{OSSS}
\item Wendy Poole, \ac{OSSS}
\item Swaroop Pophale, \ac{ORNL}
\item Sreeram Potluri, NVIDIA
\item Howard Pritchard, \ac{LANL}
\item Naveen Ravichandrasekaran, Cray Inc.
\item Michael Raymond, \ac{HPE}
\item James Ross, \ac{ARL}
\item Pavel Shamis, ARM Inc.
\item Sameer Shende, \ac{UO}
\item Lauren Smith, \ac{DoD}

\end{itemize}

\section*{Alumni Authors and Collaborators}
\begin{itemize}
\item Amrita Banerjee, \ac{UH}
\item Monika ten Bruggencate, Cray Inc.
\item Eduardo D'Azevedo, \ac{ORNL}
\item Oscar Hernandez, \ac{ORNL}
\item Gregory Koenig, \ac{ORNL}
\item Graham Lopez, \ac{ORNL}
\item Ricardo Mauricio, \ac{UH}
\item Ram Nanjegowda, \ac{UH}
\item Aaron Welch, \ac{ORNL}

\end{itemize}

\date{\today}

\section*{Acknowledgments}
The \openshmem specification belongs to Open Source Software Solutions, Inc.
(OSSS), a non-profit organization, under an agreement with HPE. For a current list
of Contributors and Collaborators, please see
  \url{http://www.openshmem.org/site/Contributors/}.
We gratefully acknowledge support from
Oak Ridge National Laboratory's
Extreme Scale Systems Center and the continuing support of the Department of Defense.\\
\\
We would also like to acknowledge the contribution of the members of the
\openshmem mailing list for their ideas, discussions, suggestions, and
constructive criticism which has helped us improve this document.\\
\\
\openshmem[1.4] is dedicated to the memory of David Charles Knaak. David was a highly involved
colleague and contributor to the entire OpenSHMEM project. He will be missed.


\setcounter{tocdepth}{4}
\setcounter{secnumdepth}{4}
\tableofcontents

\mainmatter  % included for use of documenttype 'book'

% Set header/footer for main content
\pagestyle{fancy}   %replacing {headings} with {fancy} for customization
\withlinenumbers    %adds line numbers to edges of normal pages
\fancyhf{}
\fancyhead[RE, LO]{\rightmark}
\fancyhead[RO, LE]{\thepage}
\renewcommand{\headrulewidth}{0pt}
\renewcommand{\thesection}{\arabic{section}}

{ %using setlength to force standardized spacing, if needed
% this command is ended in backmatter.tex
%\setlength{\baselineskip}{3pt plus 3pt minus 3pt}

\setlength{\parskip}{3pt}




\section{The OpenSHMEM Effort}\label{subsec:openshmem_effort}
\input{content/the_openshmem_effort}

\section{Programming Model Overview}\label{subsec:programming_model}
\input{content/programming_model_overview}

\section{Memory Model}\label{subsec:memory_model}
\begin{figure}[h]
\includegraphics[width=0.95\textwidth]{figures/mem_model}      
\caption{\OSH Memory Model}                                   
\label{fig:mem_model}                                               
\end{figure}      
%
An \openshmem program consists of data objects that are private to each \ac{PE}
and data  objects that are remotely accessible by all \ac{PE}s. Private data
objects are stored in the local memory of each \ac{PE} and can only be accessed
by the \ac{PE} itself; these data objects cannot be accessed by other \ac{PE}s
via \openshmem routines. Private data objects follow the memory model of
\Clang{} or \Fortran. Remotely accessible objects, however, can be accessed by
remote \ac{PE}s using \openshmem routines.  Remotely accessible data objects are
called \emph{Symmetric Data Objects}.  Each symmetric data object has a
corresponding object with the same name, type, and size on all PEs where that object is
accessible via the \openshmem \ac{API}\footnote{For efficiency reasons,
the same offset (from an arbitrary memory address) for symmetric data
objects might be used on all \acp{PE}. Further discussion about symmetric heap
layout and implementation efficiency can be found in section
\ref{subsec:shfree}}.  (For the definition of what is accessible, see the
descriptions for \FUNC{shmem\_pe\_accessible} and \FUNC{shmem\_addr\_accessible}
in sections \ref{subsec:shmem_pe_accessible} and
\ref{subsec:shmem_addr_accessible}.) Symmetric data objects accessed via typed
\openshmem interfaces are required to be natural aligned based on their type
requirements and underlying architecture.  In \openshmem the following kinds of
data objects are symmetric:
%
\begin{itemize}
  \item \Fortran{} data objects in common blocks or with the  SAVE  attribute.
      These data objects must not be defined in a dynamic shared object (DSO).
  \item Global and static \Clang{} and \Cpp variables. These data objects must
      not  be defined in a DSO.
  \item \Fortran{} arrays allocated with \textit{shpalloc} 
  \item \Clang{} and \Cpp data allocated by \textit{shmem\_malloc}
\end{itemize}       

\openshmem dynamic memory allocation routines (\textit{shpalloc} and
\textit{shmem\_malloc}) allow collective allocation of \emph{Symmetric Data
Objects} on a special memory region called the \emph{Symmetric Heap}. The
Symmetric Heap is created during the execution of a program at a memory location
determined by the implementation \newtext{or on multiple user determined memory 
locations called as the \emph{Symmetric Memory Partitions}}. The Symmetric 
Heap may reside in different memory regions on different \acp{PE}. 
Figure~\ref{fig:mem_model} shows  how \openshmem implements a \ac{PGAS} model 
using remotely accessible symmetric objects and private data objects when executing 
an \openshmem program. Symmetric data objects are stored on the symmetric heap or 
in the global/static memory section of each \ac{PE}. 


\section{Execution Model}\label{subsec:execution_model}
\input{content/execution_model}

\section{Language Bindings and Conformance}\label{subsec:bindings}
\input{content/language_bindings_and_conformance}

\section{Library Constants}\label{subsec:library_constants}
The constants that start with SHMEM\_* are for both \Fortran{}
and \CorCpp, and they are compile-time constants. 
All constants that start with
\_SHMEM\_* are deprecated and provided for backwards compatibility.
\newline
\newline
\begin{tabular}{|p{0.4\textwidth}|p{0.5\textwidth}|}
\hline
\textbf{Constant} & \textbf{Description}
\tabularnewline
\hline 
\hline
%new
\vspace{3mm}
\vtop{\hbox{\CorCppFor:}
\hbox{\hspace*{12mm} \const{SHMEM\_SYNC\_SIZE}}}
& Length of a work array that can be used with any SHMEM collective
communication operation. The value of this constant is implementation
specific. Refer to the individual \hyperref[subsec:coll]{Collective Routines} for more information
about the usage of this constant. Work arrays sized for specific operations may
consume less memory.\tabularnewline
%new
\hline 
\vspace{3mm}
\vtop{\hbox{\CorCppFor:} 
\hbox{\hspace*{12mm} \const{SHMEM\_BCAST\_SYNC\_SIZE}}} 
& 
Length of the \VAR{pSync} arrays needed for broadcast routines. The value
of this constant is implementation specific. Refer to the
\hyperref[subsec:shmem_broadcast]{Broadcast Routines} section under
\hyperref[sec:openshmem_library_api]{Library Routines} for more information
about the usage of this constant. \tabularnewline
\hline 
\vspace{3mm}
\vtop{\hbox{\CorCppFor:} 
\hbox{\hspace*{12mm} \const{SHMEM\_SYNC\_VALUE}}} 
& 
The value used to initialize the elements of \VAR{pSync} arrays. The
value of this constant is implementation specific.\tabularnewline
\hline
\vspace{3mm}
\vtop{\hbox{\CorCppFor:} 
\hbox{\hspace*{12mm} \const{SHMEM\_REDUCE\_SYNC\_SIZE}}}
& 
Length of the work arrays needed for reduction routines. The value
of this constant is implementation specific. Refer to the
\hyperref[subsec:shmem_reductions]{Reduction Routines} section under
\hyperref[sec:openshmem_library_api]{Library Routines} for more information
about the usage of this constant.\tabularnewline
\hline
\vspace{3mm}
\vtop{\hbox{\CorCppFor:} 
\hbox{\hspace*{12mm} \const{SHMEM\_BARRIER\_SYNC\_SIZE}}} 
& 
Length of the work array needed for barrier routines. The value
of this constant is implementation specific. Refer to the
\hyperref[subsec:shmem_barrier]{Barrier Synchronization Routines} section under
\hyperref[sec:openshmem_library_api]{Library Routines}
for more information about the usage of this constant.\tabularnewline
\hline
\vspace{3mm}
\vtop{\hbox{\CorCppFor:}
\hbox{\hspace*{12mm} \const{SHMEM\_COLLECT\_SYNC\_SIZE}}} 
& 
Length of the work array needed for collect routines. The value
of this constant is implementation specific. Refer to the
\hyperref[subsec:shmem_collect]{Collect Routines} section under
\hyperref[sec:openshmem_library_api]{Library Routines} for more information
about the usage of this constant.\tabularnewline
\hline
\vspace{3mm}
\vtop{\hbox{\CorCppFor:}
\hbox{\hspace*{12mm} \const{SHMEM\_ALLTOALL\_SYNC\_SIZE}}} 
& 
Length of the work array needed for \FUNC{shmem\_alltoall}
routines. The value of this constant is implementation
specific. Refer to the \hyperref[subsec:shmem_alltoall]{Alltoall
routines} sections under \hyperref[sec:openshmem_library_api]{Library Routines}
for more information about the usage of this constant.\tabularnewline
\hline
\end{tabular}

\begin{tabular}{|p{0.4\textwidth}|p{0.5\textwidth}|}
\hline
\vspace{3mm}
\vtop{\hbox{\CorCppFor:}
\hbox{\hspace*{12mm} \const{SHMEM\_ALLTOALLS\_SYNC\_SIZE}}} 
& 
Length of the work array needed for \FUNC{shmem\_alltoalls}
routines. The value of this constant is implementation
specific. Refer to the \hyperref[subsec:shmem_alltoalls]{Alltoalls
routines} sections under \hyperref[sec:openshmem_library_api]{Library Routines}
for more information about the usage of this constant.\tabularnewline
\hline
\vspace{3mm}
\vtop{\hbox{\CorCppFor:} 
\hbox{\hspace*{12mm} \const{SHMEM\_REDUCE\_MIN\_WRKDATA\_SIZE}}} 
& Minimum length of work arrays used in various collective routines.\tabularnewline
\hline
\vspace{3mm}
%\color{red}
%\vtop{\hbox{} 
%\hbox{\hspace*{12mm} \const{}} 
%\hbox{} 
%\hbox{\hspace*{12mm} \const{}}} 
%& \color{red}
%Ticket \#107 \tabularnewline
\vtop{\hbox{\CorCppFor:} 
\hbox{\hspace*{12mm} \const{SHMEM\_MAJOR\_VERSION}}}
& 
Integer representing the major version of \openshmem{} standard in use. \tabularnewline
\hline
\vspace{3mm}
\vtop{\hbox{\CorCppFor:} 
\hbox{\hspace*{12mm} \const{SHMEM\_MINOR\_VERSION}}}
& 
Integer representing the minor version of \openshmem{} standard in use. \tabularnewline
\hline
\vspace{3mm}
\vtop{\hbox{\CorCppFor:} 
\hbox{\hspace*{12mm} \const{SHMEM\_MAX\_NAME\_LEN}}}
&
Integer representing the length of vendor string. \tabularnewline
\hline
\vspace{3mm}
\vtop{\hbox{\CorCppFor:} 
\hbox{\hspace*{12mm} \const{SHMEM\_VENDOR\_STRING}}} 
&
String representing the vendor name of length less than
\const{SHMEM\_MAX\_NAME\_LEN}.  In Fortran the string must be \const{SHMEM\_MAX\_NAME\_LEN} 
and whitespace padded.  It can also be equal in length to \const{SHMEM\_MAX\_NAME\_LEN} 
since Fortran does not NULL terminate strings. \tabularnewline
\hline
\vspace{3mm}
\vtop{\hbox{\newtext{\CorCpp:}} 
\hbox{\hspace*{12mm} \newtext{\const{SHMEM\_MAX\_PARTITIONS}}}}
&\newtext{
Integer representing the implementations maximum for the number of partitions
that can be defined by the users.} %Implementations value must be at least 7 so
%that a portable program can assume at least that possible partitions}
\tabularnewline
\hline
\vspace{3mm}
\vtop{\hbox{\newtext{\CorCpp:}}
\hbox{\hspace*{12mm} \newtext{\const{SHMEM\_MAX\_PARTITION\_ID}}}}
&\newtext{
Integer representing the implementations maximum number that can be used to
represent the partition ID.} %The value should be at least 127} 
\tabularnewline
\hline
\end{tabular}
\color{black}


\section{Library Handles}\label{subsec:library_handles}
\input{content/library_handles}

\section{Environment Variables }\label{subsec:environment_variables}
The \openshmem specification provides a set of environment variables that allows
users to configure the \openshmem implementation, and receive information about
the implementation. The implementations of the specification are free to define
additional variables. Currently, the specification defines five environment
variables. All environment variables that start with \VAR{SMA\_*} are
deprecated, but currently supported for backwards compatibility.

\medskip{}

\begin{tabular}{|l|l|l|}
\hline 
Variable & Value & Purpose\tabularnewline
\hline 
\hline 
\texttt{SHMEM\_VERSION} & any & print the library version at
start-up\tabularnewline
\hline 
\texttt{SHMEM\_INFO} & any & print helpful text about all these environment
variables\tabularnewline
\hline 
\texttt{SHMEM\_SYMMETRIC\_SIZE} & non-negative integer & number of bytes to
allocate for symmetric heap\tabularnewline
\hline 
\texttt{\newtext{SHMEM\_SYMMETRIC\_PARTITION}} 
& \newtext{Refer Section \ref{subsubsec:usr_defn_env}} 
& \newtext{Refer Section \ref{subsubsec:usr_defn_env}}\tabularnewline
\hline 
\texttt{SHMEM\_DEBUG} & any & enable debugging messages\tabularnewline
\hline 
\end{tabular}

\medskip{}




\clearpage



\section{OpenSHMEM Library API}\label{sec:openshmem_library_api}

\subsection{Library Setup, Exit, and Query Routines}
The library setup and query interfaces that initialize and monitor the parallel
environment of the \acp{PE}.

\subsubsection{\textbf{SHMEM\_INIT}}\label{subsec:shmem_init}
\input{content/shmem_init}

\subsubsection{\textbf{SHMEM\_MY\_PE}}\label{subsec:shmem_my_pe}
\input{content/shmem_my_pe}

\subsubsection{\textbf{SHMEM\_N\_PES}}\label{subsec:shmem_n_pes}
\input{content/shmem_n_pes}

\subsubsection{\textbf{SHMEM\_FINALIZE}}\label{subsec:shmem_finalize}
\input{content/shmem_finalize}

\subsubsection{\textbf{SHMEM\_GLOBAL\_EXIT}}\label{subsec:shmem_global_exit}
\input{content/shmem_global_exit}

\subsubsection{\textbf{SHMEM\_PE\_ACCESSIBLE}}\label{subsec:shmem_pe_accessible}
\input{content/shmem_pe_accessible}

\subsubsection{\textbf{SHMEM\_ADDR\_ACCESSIBLE}}\label{subsec:shmem_addr_accessible}
\apisummary{
    Determines whether an address is accessible via OpenSHMEM data transfer
    routines from the specified  remote \ac{PE}.
}

\begin{apidefinition}

\begin{Csynopsis}
int shmem_addr_accessible(const void *addr, int pe);
\end{Csynopsis}

\begin{Fsynopsis}
LOGICAL LOG, SHMEM_ADDR_ACCESSIBLE
INTEGER pe
LOG = SHMEM_ADDR_ACCESSIBLE(addr, pe)
\end{Fsynopsis}

\begin{apiarguments}
    \apiargument{IN}{addr}{Data object on the local \ac{PE}.}
    \apiargument{IN}{pe}{Integer id of a remote \ac{PE}.}
\end{apiarguments}

\apidescription{
    \FUNC{shmem\_addr\_accessible} is a query routine that indicates whether a local
    address is accessible via \openshmem routines from the specified remote \ac{PE}.

    This routine verifies that the data object is symmetric and accessible with
    respect to a remote \ac{PE} via \openshmem  data  transfer routines.  The
    specified address \VAR{addr} is a data object on the local \ac{PE}.

    This routine may be particularly useful for hybrid programming with other
    communication libraries (such as \ac{MPI}) or parallel languages.  For
    example, in SGI Altix series systems, for multiple executable MPI programs that
    use \openshmem routines, it is important to note that static memory, such as a
    \Fortran{} common block or \Clang{} global variable, is symmetric between
    processes running from the same executable file, but is not symmetric between
    processes running from different executable files.  Data allocated from the
    symmetric heap (\FUNC{shmem\_malloc}, \newtext{\FUNC{shmem\_partition\_malloc}}
    or \FUNC{shpalloc}) is symmetric across the same or different executable files.
}

\apireturnvalues{
    \CorCpp: The  return value is \CONST{1} if \VAR{addr} is a symmetric data object
    and accessible via \openshmem routines from the specified remote \ac{PE};
    otherwise, it is \CONST{0}.

    \Fortran: The return value is \CONST{.TRUE.} if \VAR{addr} is a symmetric data
    object and accessible via \openshmem routines from the specified remote \ac{PE};
    otherwise, it is \CONST{.FALSE.}.
}

\apinotes{
    None.
}

\end{apidefinition}


\subsubsection{\textbf{SHMEM\_PTR}}\label{subsec:shmem_ptr}
\input{content/shmem_ptr}

\subsubsection{\textbf{SHMEM\_INFO\_GET\_VERSION}}\label{subsec:shmem_info_get_version}
\input{content/shmem_info_get_version}

\subsubsection{\textbf{SHMEM\_INFO\_GET\_NAME}}\label{subsec:shmem_info_get_name}
\input{content/shmem_info_get_name}

\subsubsection{\textbf{START\_PES}}\label{subsec:start_pes}
\input{content/start_pes}

\subsection{Thread Support}
\label{subsec:thread_support}
\input{content/threads_intro.tex}

\subsubsection{\textbf{SHMEM\_INIT\_THREAD}}
\label{subsec:shmem_init_thread}
\input{content/shmem_init_thread}

\subsubsection{\textbf{SHMEM\_QUERY\_THREAD}}
\label{subsec:shmem_query_thread}
\input{content/shmem_query_thread}


\subsection{Memory Management Routines}
\label{sec:memory_management}

\openshmem provides a set of \acp{API} for managing the symmetric heap. The
\acp{API} allow one to dynamically allocate, deallocate, reallocate and align
symmetric data objects in the symmetric heap.

\subsubsection{\textbf{SHMEM\_MALLOC, SHMEM\_FREE, SHMEM\_REALLOC, SHMEM\_ALIGN}}\label{subsec:shfree}
\apisummary{
    Symmetric heap memory management routines.
}

\begin{apidefinition}

\begin{Csynopsis}
void *shmem_malloc(size_t size);
void shmem_free(void *ptr);
void *shmem_realloc(void *ptr, size_t size);
void *shmem_align(size_t alignment, size_t size);
\end{Csynopsis}

\begin{apiarguments}
    \apiargument{IN}{size}{The size, in bytes, of a block to be
        allocated from the symmetric heap. This argument is of type \VAR{size\_t}}
    \apiargument{IN}{ptr}{Points to a block within the symmetric heap.}
    \apiargument{IN}{alignment}{Byte alignment of the block allocated from the
        symmetric heap.}
\end{apiarguments}


\apidescription{
    The \FUNC{shmem\_malloc} routine returns a pointer to a block of at least
    \VAR{size} bytes suitably aligned for any use.  This space is allocated from the
    symmetric heap (in contrast to \FUNC{malloc}, which allocates from the private
    heap).
    
    The \FUNC{shmem\_align} routine allocates a block in the symmetric heap that has
    a byte alignment specified by the alignment argument.
    
    The \FUNC{shmem\_free} routine causes the block to which \VAR{ptr} points to be
    deallocated, that is, made available for further allocation.  If \VAR{ptr} is a
    null pointer, no action occurs. 
           
    The \FUNC{shmem\_realloc} routine changes the size of the block to which
    \VAR{ptr} points to the size (in bytes) specified by \VAR{size}.  The contents
    of the block are unchanged up to the lesser of the new and old sizes. If the new
    size is larger, the newly allocated portion of the block is
    uninitialized.  If \VAR{ptr} is a \CONST{NULL} pointer, the
    \FUNC{shmem\_realloc} routine behaves like the \FUNC{shmem\_malloc} routine for
    the specified size.  If \VAR{size} is \CONST{0} and \VAR{ptr} is not a
    \CONST{NULL} pointer, the block to which it points is freed. If the space cannot
    be allocated, the block to which \VAR{ptr} points is unchanged.
    
    The \FUNC{shmem\_malloc}, \FUNC{shmem\_free}, and \FUNC{shmem\_realloc} routines
    are provided  so that multiple \ac{PE}s in a program can allocate symmetric,
    remotely accessible memory blocks.  These memory blocks can then be used with
    \openshmem communication routines.  Each of these routines call the
    \FUNC{shmem\_barrier\_all} routine before returning; this ensures that all
    \ac{PE}s participate in the memory allocation, and that the memory on other
    \ac{PE}s can be used as  soon as the local \ac{PE} returns.  The user is
    responsible for calling these routines with identical argument(s) on all
    \ac{PE}s; if differing \VAR{size} arguments are used, the behavior of the call
    and any subsequent \openshmem calls becomes undefined.
}

\apireturnvalues{
    The \FUNC{shmem\_malloc} routine returns a pointer to the allocated space;
    otherwise, it returns a \CONST{NULL} pointer.
    
    The \FUNC{shmem\_free} routine returns no value.
    
    The \FUNC{shmem\_realloc} routine returns a pointer to the allocated space
    (which may have moved); otherwise, it returns a null pointer.
    
    The \FUNC{shmem\_align} routine returns an aligned pointer to the allocated
    space; otherwise, it returns a \CONST{NULL} pointer.
}

\apinotes{ 
    As of Specification 1.2 the use of \FUNC{shmalloc}, \FUNC{shmemalign},
    \FUNC{shfree},  and \FUNC{shrealloc} has been deprecated. Although OpenSHMEM
    libraries are required to support the calls, program users are encouraged to use
    \FUNC{shmem\_malloc}, \FUNC{shmem\_align}, \FUNC{shmem\_free}, and
    \FUNC{shmem\_realloc} instead.  The behavior and signature  of the routines
    remains unchanged from the deprecated versions.
    					 
    \newtext{Usage of \FUNC{shmem\_malloc} and \FUNC{shmem\_align} along with
    Symmetric Memory Partitions would result in the memory space to be
    allocated on the symmetric heap created on partition with ID as 1.}

    The total size of the symmetric heap is determined at job startup.  One can
    adjust the size of the heap using the \CONST{SHMEM\_SYMMETRIC\_SIZE} or
    \CONST{SHMEM\_SYMMETRIC\_PARTITION} environment variable (where available).	
    
    The \FUNC{shmem\_malloc}, \FUNC{shmem\_free}, and \FUNC{shmem\_realloc} routines
    differ from the private heap allocation routines in that all \ac{PE}s in a
    program must call them (a barrier is used to ensure this).
}		

\apiimpnotes{
    The symmetric heap allocation routines always return a pointer to corresponding
    symmetric objects across all PEs. The \openshmem{} specification does not
    require that the virtual addresses are equal across all \acp{PE}. Nevertheless,
    the implementation must avoid costly address translation operations in the
    communication path, including order $N$ (where $N$ is the number of \acp{PE})
    memory translation tables.  In order to avoid address translations, the
    implementation may re-map the allocated block of memory based on agreed virtual
    address.  Additionally, some operating systems provide an option to disable
    virtual address randomization, which enables predictable allocation of virtual
    memory addresses.
}

\end{apidefinition}


\subsubsection{\textbf{SHMEM\_CALLOC}}\label{subsec:shmem_calloc}
\input{content/shmem_calloc.tex}

\subsubsection{\textbf{SHPALLOC}}\label{subsec:shpalloc}
\input{content/shpalloc.tex}

\subsubsection{\textbf{SHPCLMOVE}}\label{subsec:shpclmove}
\input{content/shpclmove.tex}

\subsubsection{\textbf{SHPDEALLC}}\label{subsec:shpdeallc}
\input{content/shpdeallc.tex}


\subsection{Communication Management Routines}
\label{sec:ctx}
All \openshmem \ac{RMA}, \ac{AMO}, and memory ordering routines must be
performed on a valid communication context.  The communication context defines an
independent ordering and completion environment, allowing users to manage the
overlap of communication with computation and also to manage communication
operations performed by separate threads within a multithreaded \ac{PE}.  For
example, in single-threaded environments, contexts may be used to pipeline
communication and computation.  In multithreaded environments, contexts may
additionally provide thread isolation, eliminating overheads resulting from
thread interference.

A specific communication context is referenced through a context handle, which is
passed as an argument in the \Cstd \CTYPE{shmem\_ctx\_*} and type-generic \ac{API}
routines.  \ac{API} routines that do not accept a context handle argument operate on the
default context.  The default context can be used explicitly through the
\LibHandleRef{SHMEM\_CTX\_DEFAULT} handle.
Context handles are of type \CTYPE{shmem\_ctx\_t} and may be used for
language-level assignment and equality comparison.

The default context is valid for the duration of the \openshmem portion of
an application.
Contexts created by a successful call to \FUNC{shmem\_ctx\_create} remain
valid until they are destroyed.
A handle value that does not correspond to a valid context is considered
to be invalid, and its use in \ac{RMA} and \ac{AMO} routines results in
undefined behavior.
A context handle may be initialized to or assigned the value
\CONST{SHMEM\_CTX\_INVALID} to indicate that handle does not reference a
valid communication context.
When managed in this way, applications can use an equality comparison
to test whether a given context handle references a valid context.

\subsubsection{\textbf{SHMEM\_CTX\_CREATE}}
\label{subsec:shmem_ctx_create}
\input{content/shmem_ctx_create.tex}

\subsubsection{\textbf{SHMEM\_CTX\_DESTROY}}
\label{subsec:shmem_ctx_destroy}
\input{content/shmem_ctx_destroy.tex}


\subsection{Remote Memory Access Routines}\label{sec:rma}
\input{content/rma_intro.tex}

\subsubsection{\textbf{SHMEM\_PUT}}\label{subsec:shmem_put}
\input{content/shmem_put.tex}

\subsubsection{\textbf{SHMEM\_P}}\label{subsec:shmem_p}
\input{content/shmem_p.tex}

\subsubsection{\textbf{SHMEM\_IPUT}}\label{subsec:shmem_iput}
\input{content/shmem_iput.tex}

\subsubsection{\textbf{SHMEM\_GET}}\label{subsec:shmem_get}
\input{content/shmem_get.tex}

\subsubsection{\textbf{SHMEM\_G}}\label{subsec:shmem_g}
\input{content/shmem_g.tex}

\subsubsection{\textbf{SHMEM\_IGET}}\label{subsec:shmem_iget}
\input{content/shmem_iget.tex}


\subsection{Non-blocking Remote Memory Access Routines}\label{sec:rma_nbi}

\subsubsection{\textbf{SHMEM\_PUT\_NBI}}\label{subsec:shmem_put_nbi}
\input{content/shmem_put_nbi.tex}

\subsubsection{\textbf{SHMEM\_PUT\_SIGNAL\_NBI}}\label{subsec:shmem_put_signal_nbi}
\color{Green}
\apisummary{
    The nonblocking put-with-signal routines provide a method for copying data
    from a contiguous local data object to a data object on a specified \ac{PE}
    and subsequently setting a remote flag to signal completion.
}

\begin{apidefinition}

\begin{C11synopsis}
void @\FuncDecl{shmem\_put\_signal\_nbi}@(TYPE *dest, const TYPE *source, size_t nelems, uint64_t *restrict sig_addr, uint64_t signal, int pe);
void @\FuncDecl{shmem\_put\_signal\_nbi}@(shmem_ctx_t ctx, TYPE *dest, const TYPE *source, size_t nelems, uint64_t *restrict sig_addr, uint64_t signal, int pe);
\end{C11synopsis}
where \TYPE{} is one of the standard \ac{RMA} types specified by Table \ref{stdrmatypes}.

\begin{Csynopsis}
void @\FuncDecl{shmem\_\FuncParam{TYPENAME}\_put\_signal\_nbi}@(TYPE *dest, const TYPE *source, size_t nelems, uint64_t *restrict sig_addr, uint64_t signal, int pe);
void @\FuncDecl{shmem\_ctx\_\FuncParam{TYPENAME}\_put\_signal\_nbi}@(shmem_ctx_t ctx, TYPE *dest, const TYPE *source, size_t nelems, uint64_t *restrict sig_addr, uint64_t signal, int pe);
\end{Csynopsis}
where \TYPE{} is one of the standard \ac{RMA} types and has a corresponding \TYPENAME{} specified by Table \ref{stdrmatypes}.

\begin{CsynopsisCol}
void @\FuncDecl{shmem\_put\FuncParam{SIZE}\_signal\_nbi}@(void *dest, const void *source, size_t nelems, uint64_t *restrict sig_addr, uint64_t signal, int pe);
void @\FuncDecl{shmem\_ctx\_put\FuncParam{SIZE}\_signal\_nbi}@(shmem_ctx_t ctx, void *dest, const void *source, size_t nelems, uint64_t *restrict sig_addr, uint64_t signal, int pe);
\end{CsynopsisCol}
where \SIZE{} is one of \CONST{8, 16, 32, 64, 128}.

\begin{CsynopsisCol}
void @\FuncDecl{shmem\_putmem\_signal\_nbi}@(void *dest, const void *source, size_t nelems, uint64_t *restrict sig_addr, uint64_t signal, int pe);
void @\FuncDecl{shmem\_ctx\_putmem\_signal\_nbi}@(shmem_ctx_t ctx, void *dest, const void *source, size_t nelems, uint64_t *restrict sig_addr, uint64_t signal, int pe);
\end{CsynopsisCol}

\begin{apiarguments}
    \apiargument{IN}{ctx}{The context on which to perform the operation.
      When this argument is not provided, the operation is performed on
      \CONST{SHMEM\_CTX\_DEFAULT}.}
    \apiargument{OUT}{dest}{Data object to be updated on the remote \ac{PE}. This
    data object must be remotely accessible.}
    \apiargument{IN}{source}{Data object containing the data to be copied.}
    \apiargument{IN}{nelems}{Number of elements in the \VAR{dest} and \VAR{source}
    arrays. \VAR{nelems} must be of type \VAR{size\_t} for \Cstd.}
    \apiargument{OUT}{sig\_addr}{Data object to be updated on the remote
    \ac{PE} as the signal. This signal data object must be
    remotely accessible.}
    \apiargument{IN}{signal}{Unsigned 64-bit value that is assigned to the
    remote \VAR{sig\_addr} signal data object.}
    \apiargument{IN}{pe}{\ac{PE} number of the remote \ac{PE}.}
\end{apiarguments}

\apidescription{
    The routines return after posting the operation. The operation is considered
    complete after the subsequent call to \FUNC{shmem\_quiet}. At the completion
    of \FUNC{shmem\_quiet}, the data has been copied out of the \VAR{source}
    array on the local \ac{PE} and delivered into the \VAR{dest} array on the
    destination \ac{PE}. The delivery of \VAR{signal} flag on the remote
    \ac{PE} indicates the delivery of its corresponding \VAR{dest} data words
    into the data object on the remote \ac{PE}.
}

\apireturnvalues{
    None.
}

\apinotes{
    The \VAR{dest} and \VAR{sig\_addr} data objects must both be remotely
    accessible. The \VAR{sig\_addr} and \VAR{dest} could be of different kinds,
    for example, one could be a global/static \Cstd variable and the other could
    be allocated on the symmetric heap.

    The restrict qualifier in \VAR{sig\_addr} expects the data object to be
    distinct from \VAR{dest} and \VAR{source} data objects.

    The delivery of \VAR{signal} flag on the remote \ac{PE} indicates only the
    delivery of its corresponding \VAR{dest} data words into the data object on
    the remote \ac{PE}. Without a memory-ordering operation, there is no implied
    ordering between the delivery of the signal word of a nonblocking
    put-with-signal routine and another data transfer. For example, the delivery
    of the signal word in a sequence consisting of a put routine followed by a
    nonblocking put-with-signal routine does not imply delivery of the put
    routine's data.
}

\end{apidefinition}
\color{Black}


\subsubsection{\textbf{SHMEM\_GET\_NBI}}\label{subsec:shmem_get_nbi}
\input{content/shmem_get_nbi.tex}


\subsection{Atomic Memory Operations}\label{sec:amo}
\input{content/atomics_intro}

\subsubsection{\textbf{SHMEM\_ATOMIC\_FETCH}}
\label{subsec:shmem_atomic_fetch}
\input{content/shmem_atomic_fetch.tex}

\subsubsection{\textbf{SHMEM\_ATOMIC\_SET}}
\label{subsec:shmem_atomic_set}
\input{content/shmem_atomic_set.tex}

\subsubsection{\textbf{SHMEM\_ATOMIC\_COMPARE\_SWAP}}
\label{subsec:shmem_atomic_compare_swap}
\input{content/shmem_atomic_compare_swap.tex}

\subsubsection{\textbf{SHMEM\_ATOMIC\_SWAP}}
\label{subsec:shmem_atomic_swap}
\input{content/shmem_atomic_swap.tex}

\subsubsection{\textbf{SHMEM\_ATOMIC\_FETCH\_INC}}
\label{subsec:shmem_atomic_fetch_inc}
\input{content/shmem_atomic_fetch_inc.tex}

\subsubsection{\textbf{SHMEM\_ATOMIC\_INC}}
\label{subsec:shmem_atomic_inc}
\input{content/shmem_atomic_inc.tex}

\subsubsection{\textbf{SHMEM\_ATOMIC\_FETCH\_ADD}}
\label{subsec:shmem_atomic_fetch_add}
\input{content/shmem_atomic_fetch_add.tex}

\subsubsection{\textbf{SHMEM\_ATOMIC\_ADD}}
\label{subsec:shmem_atomic_add}
\input{content/shmem_atomic_add.tex}

\subsubsection{\textbf{SHMEM\_ATOMIC\_FETCH\_AND}}
\label{subsec:shmem_atomic_fetch_and}
\input{content/shmem_atomic_fetch_and.tex}

\subsubsection{\textbf{SHMEM\_ATOMIC\_AND}}
\label{subsec:shmem_atomic_and}
\input{content/shmem_atomic_and.tex}

\subsubsection{\textbf{SHMEM\_ATOMIC\_FETCH\_OR}}
\label{subsec:shmem_atomic_fetch_or}
\input{content/shmem_atomic_fetch_or.tex}

\subsubsection{\textbf{SHMEM\_ATOMIC\_OR}}
\label{subsec:shmem_atomic_or}
\input{content/shmem_atomic_or.tex}

\subsubsection{\textbf{SHMEM\_ATOMIC\_FETCH\_XOR}}
\label{subsec:shmem_atomic_fetch_xor}
\input{content/shmem_atomic_fetch_xor.tex}

\subsubsection{\textbf{SHMEM\_ATOMIC\_XOR}}
\label{subsec:shmem_atomic_xor}
\input{content/shmem_atomic_xor.tex}





\subsection{Collective Routines}\label{subsec:coll}
\input{content/collective_intro.tex}

\subsubsection{\textbf{SHMEM\_BARRIER\_ALL}}\label{subsec:shmem_barrier_all}
\input{content/shmem_barrier_all.tex}

\subsubsection{\textbf{SHMEM\_BARRIER}}\label{subsec:shmem_barrier}
\input{content/shmem_barrier.tex}

\subsubsection{\textbf{SHMEM\_SYNC\_ALL}}\label{subsec:shmem_sync_all}
\input{content/shmem_sync_all.tex}

\subsubsection{\textbf{SHMEM\_SYNC}}\label{subsec:shmem_sync}
\input{content/shmem_sync.tex}

\subsubsection{\textbf{SHMEM\_BROADCAST}}\label{subsec:shmem_broadcast}
\input{content/shmem_broadcast.tex}

\subsubsection{\textbf{SHMEM\_COLLECT, SHMEM\_FCOLLECT}}\label{subsec:shmem_collect}
\input{content/shmem_collect.tex}

\subsubsection{\textbf{SHMEM\_REDUCTIONS}}\label{subsec:shmem_reductions}
\input{content/shmem_reductions.tex}

\subsubsection{\textbf{SHMEM\_ALLTOALL}}\label{subsec:shmem_alltoall}
\input{content/shmem_alltoall.tex}

\subsubsection{\textbf{SHMEM\_ALLTOALLS}}\label{subsec:shmem_alltoalls}
\input{content/shmem_alltoalls.tex}





\subsection{Point-To-Point Synchronization Routines}\label{subsec:p2p_intro}
The following section discusses \openshmem \acp{API} that provide a mechanism
for synchronization between two \acp{PE} based on the value of a symmetric data
object.
The point-to-point synchronization routines can be used to portably ensure
that memory access operations observe remote updates in the order enforced by
the initiator \ac{PE} using the put-with-signal(refer
section~\ref{subsec:shmem_put_signal}, \FUNC{shmem\_fence} and
\FUNC{shmem\_quiet} routines.

Where appropriate compiler support is available, \openshmem provides
type-generic point-to-point synchronization interfaces via \Cstd[11] generic
selection. Such type-generic routines are supported for the
``point-to-point synchronization types'' identified in
Table~\ref{p2psynctypes}.

The point-to-point synchronization types include some of the exact-width
integer types defined in \HEADER{stdint.h} by \Cstd[99]~\S7.18.1.1 and
\Cstd[11]~\S7.20.1.1. When the \Cstd translation environment
does not provide exact-width integer types with \HEADER{stdint.h}, an
\openshmem implemementation is not required to provide support for these types.

\begin{table}[h]
  \begin{center}
    \begin{tabular}{|l|l|}
      \hline
      \TYPE              & \TYPENAME  \\ \hline
      short              & short      \\ \hline
      int                & int        \\ \hline
      long               & long       \\ \hline
      long long          & longlong   \\ \hline
      unsigned short     & ushort     \\ \hline
      unsigned int       & uint       \\ \hline
      unsigned long      & ulong      \\ \hline
      unsigned long long & ulonglong  \\ \hline
      int32\_t           & int32      \\ \hline
      int64\_t           & int64      \\ \hline
      uint32\_t          & uint32     \\ \hline
      uint64\_t          & uint64     \\ \hline
      size\_t            & size       \\ \hline
      ptrdiff\_t         & ptrdiff    \\ \hline
    \end{tabular}
    \TableCaptionRef{Point-to-Point Synchronization Types and Names}
    \label{p2psynctypes}
  \end{center}
\end{table}

The point-to-point synchronization interface provides named constants whose
values are integer constant expressions that specify the comparison operators
used by \openshmem synchronization routines.
The constant names and associated operations are
presented in Table~\ref{p2p-consts}.  For Fortran, the constant names of
Table~\ref{p2p-consts} shall be identifiers for integer parameters of
default kind corresponding to the associated comparison operation.

\begin{table}[h]
  \begin{center}
    \begin{tabular}{ll}
      \hline
      Constant Name                 & Comparison               \\ \hline
      \LibConstRef{SHMEM\_CMP\_EQ}  & Equal                    \\
      \LibConstRef{SHMEM\_CMP\_NE}  & Not equal                \\
      \LibConstRef{SHMEM\_CMP\_GT}  & Greater than             \\
      \LibConstRef{SHMEM\_CMP\_GE}  & Greater than or equal to \\
      \LibConstRef{SHMEM\_CMP\_LT}  & Less than                \\
      \LibConstRef{SHMEM\_CMP\_LE}  & Less than or equal to    \\ \hline
    \end{tabular}
    \TableCaptionRef{Point-to-Point Comparison Constants}
    \label{p2p-consts}
  \end{center}
\end{table}


\subsubsection{\textbf{SHMEM\_WAIT\_UNTIL}}\label{subsec:shmem_wait_until}
\input{content/shmem_wait_until.tex}

\subsubsection{\textbf{SHMEM\_TEST}}\label{subsec:shmem_test}
\input{content/shmem_test.tex}





\subsection{Memory Ordering Routines}\label{subsec:memory_order}
The following section discusses \openshmem \acp{API} that provide mechanisms to
ensure ordering and/or delivery of \OPR{Put}, \ac{AMO}, memory store,
and non-blocking \PUT{} and \GET{} routines to symmetric data objects.

\subsubsection{\textbf{SHMEM\_FENCE}}\label{subsec:shmem_fence}
\input{content/shmem_fence.tex}

\subsubsection{\textbf{SHMEM\_QUIET}}\label{subsec:shmem_quiet}
\input{content/shmem_quiet.tex}

\subsubsection{Synchronization and Communication Ordering in OpenSHMEM}
\input{content/synchronization_model.tex}






\subsection{Distributed Locking Routines}
The following section discusses \openshmem locks as a mechanism to provide
mutual exclusion. Three routines are available for distributed locking,
\textit{set, test} and \textit{clear}.

\subsubsection{\textbf{SHMEM\_LOCK}}\label{subsec:shmem_lock}
\input{content/shmem_lock.tex}





\subsection{Cache Management}
All of these routines are deprecated and are provided for backwards
compatibility.  Implementations must include all items in this section, and the
routines should function properly and may notify the user about deprecation of
their use.

\subsubsection{\textbf{SHMEM\_CACHE}}\label{subsec:shmem_cache}
\input{content/shmem_cache.tex}

\clearpage
\clearpage %%%%%%%%%%%%%%%%%%%%%%%%%%%%%%%%%%%%%%%%%%%%%%%%%%%%%%%%%%%%

\appendix

%defining pagestyle for annex
%\pagestyle{plain} \withlinenumbers
\pagestyle{fancy} \withlinenumbers
\fancyhf{}
\fancyhead[RE, LO]{\leftmark}
\fancyhead[RO, LE]{\thepage}
\fancyfoot[CE, CO]{\thepage}
\renewcommand{\headrulewidth}{0pt}




\chapter{Writing OpenSHMEM Programs}
\section*{Incorporating OpenSHMEM into Programs}\label{sec:writing_programs}

The following section describes how to write a ``Hello World" \openshmem program.
To write a ``Hello World" \openshmem program, the user must:

\begin{itemize}
\item Include the header file \HEADER{shmem.h} for \Cstd or \HEADER{shmem.fh} for \Fortran.
\item Add the initialization call \hyperref[subsec:shmem_init]{\FUNC{shmem\_init}}.
\item Use \openshmem calls to query the local \ac{PE} number
    (\hyperref[subsec:shmem_my_pe]{\FUNC{shmem\_my\_pe}}) and the total number
    of \acp{PE} (\hyperref[subsec:shmem_n_pes]{\FUNC{shmem\_n\_pes}}).
\item Add the finalization call \hyperref[subsec:shmem_finalize]{\FUNC{shmem\_finalize}}.
\end{itemize}

In \openshmem, the order in which lines appear in the output is not
deterministic because \acp{PE} execute asynchronously in parallel.

\begin{minipage}{\linewidth}
\vspace{0.1in}
\numberedlisting{caption={``Hello World'' example program in \Cstd},label=openshmem-hello,language=OSH2+C}
                {example_code/hello-openshmem.c}
\outputlisting{language=bash,caption={Possible ordering of expected output with 4 \acp{PE} from the program in Listing~\ref{openshmem-hello}}}
                {example_code/hello-openshmem-c.output}
\vspace{0.1in}
\end{minipage}

\clearpage %%%%%%%%%%%%%%%%%%%%%%%%%%%%%%%%%%%%%%%%%%%%%%%%%%%%%%%%%%%%

\begin{deprecate}
\openshmem also provides a \Fortran API. Listing~\ref{openshmem-hello-f90} shows a similar program written in \Fortran.

\begin{minipage}{\linewidth}
\vspace{0.1in}
\numberedlisting{caption={``Hello World'' example program in \Fortran},label=openshmem-hello-f90,language=OSH2+F}
                {example_code/hello-openshmem.f90}
\outputlisting{language=bash,caption={Possible ordering of expected output with 4 \acp{PE} from the program in Listing~\ref{openshmem-hello-f90}}}
                {example_code/hello-openshmem-f90.output}
\vspace{0.1in}
\end{minipage}
\end{deprecate}

\clearpage %%%%%%%%%%%%%%%%%%%%%%%%%%%%%%%%%%%%%%%%%%%%%%%%%%%%%%%%%%%%

The example in Listing~\ref{openshmem-hello-symmetric} shows a more complex
\openshmem program that illustrates the use of symmetric data objects.
Note the declaration of the \VAR{static short dest} array and its use as the
remote destination in \hyperref[subsec:shmem_put]{\FUNC{shmem\_put}}.

The \KEYWORD{static} keyword makes the \VAR{dest} array symmetric on all \acp{PE}.
Each \ac{PE} is able to transfer data to a remote \dest{} array by simply
specifying to an OpenSHMEM routine such as \hyperref[subsec:shmem_put]{\FUNC{shmem\_put}}
the local address of the symmetric data object that will receive the data.
This local address resolution aids programmability because the address of the
\dest{} need not be exchanged with the active side (\ac{PE} \CONST{0}) prior to
the \acf{RMA} routine.

Conversely, the declaration of the \VAR{short source} array is asymmetric
(local only).
The \source{} object does not need to be symmetric because \PUT{} handles the
references to the \VAR{source} array only on the active (local) side.

\begin{minipage}{\linewidth}
\vspace{0.1in}
\numberedlisting{caption={Example program with symmetric data objects},label=openshmem-hello-symmetric,language=OSH2+C}
                {example_code/writing_shmem_example.c}
\outputlisting{language=bash,caption={Possible ordering of expected output with 4 \acp{PE} from the program in Listing~\ref{openshmem-hello-symmetric}}}
                {example_code/writing_shmem_example.output}
\vspace{0.1in}
\end{minipage}




\chapter{Compiling and Running Programs}\label{sec:compiling}
The \openshmem Specification does not specify how
\openshmem programs are compiled, linked, and run. This section shows some
examples of how wrapper programs are utilized in the \openshmem Reference
Implementation to compile and launch programs.

\section{Compilation}
\subsection*{Programs written in \Cstd}

The \openshmem Reference Implementation provides a wrapper program, named
\textbf{oshcc}, to aid in the compilation of \Cstd programs.
The wrapper may be called as follows:

\begin{lstlisting}[language=bash]
oshcc <compiler options> -o myprogram myprogram.c
\end{lstlisting}
Where the $\langle\mbox{compiler options}\rangle$ are options understood by the
underlying \Cstd compiler called by \textbf{oshcc}.


\subsection*{Programs written in \Cpp}

The \openshmem Reference Implementation provides a wrapper program, named
\textbf{oshc++}, to aid in the compilation of \Cpp programs.
The wrapper may be called as follows:

\begin{lstlisting}[language=bash]
oshc++ <compiler options> -o myprogram myprogram.cpp
\end{lstlisting}
Where the $\langle\mbox{compiler options}\rangle$ are options understood by the
underlying \Cpp compiler called by \textbf{oshc++}.


\subsection*{Programs written in \Fortran}

\begin{deprecate}
The \openshmem Reference Implementation provides a wrapper program, named
\textbf{oshfort}, to aid in the compilation of \Fortran programs.
The wrapper may be called as follows:

\begin{lstlisting}[language=bash]
oshfort <compiler options> -o myprogram myprogram.f
\end{lstlisting}
Where the $\langle\mbox{compiler options}\rangle$ are options understood by the
underlying \Fortran compiler called by \textbf{oshfort}.
\end{deprecate}

\section{Running Programs}

The \openshmem Reference Implementation provides a wrapper program, named
\textbf{oshrun}, to launch \openshmem programs.
The wrapper may be called as follows:

\begin{lstlisting}[language=bash]
oshrun <runner options> -np <#> <program> <program arguments>
\end{lstlisting}
The arguments for \textbf{oshrun} are:

\begin{tabular}{p{0.3\textwidth}p{0.6\textwidth}}
$\langle\mbox{runner options}\rangle$ & {Options passed to the underlying launcher.}\tabularnewline
-np $\langle\mbox{\#}\rangle$ & {The number of \acp{PE} to be used in the execution.}\tabularnewline
$\langle\mbox{program}\rangle$ & {The program executable to be launched.}\tabularnewline
$\langle\mbox{program arguments}\rangle$ & {Flags and other parameters to pass to the program.}\tabularnewline
\end{tabular}




\chapter{Undefined Behavior in OpenSHMEM}\label{sec:undefined}

The \openshmem Specification formalizes the expected behavior of
its library routines.  In cases where routines are improperly used
or the input is not in accordance with the Specification, the behavior
is undefined.

\begin{longtable}{|>{\raggedright}p{0.3\textwidth}|>{\raggedright}p{0.6\textwidth}|}
\hline
\textbf{Inappropriate Usage} & \textbf{Undefined Behavior}\tabularnewline
\hline
\endhead
Uninitialized library & If the \openshmem library is not initialized,
calls to non-initializing \openshmem routines have undefined
behavior.  For example, an implementation may try to continue or may abort
immediately upon an \openshmem call into the uninitialized library.
\tabularnewline
\hline
Multiple calls to initialization routines & In an \openshmem program where
the initialization routines \FUNC{shmem\_init} or \FUNC{shmem\_init\_thread}
have already been called, any subsequent calls to these initialization routines
result in undefined behavior.
\tabularnewline
\hline
Accessing non-existent \acp{PE} & If a communications routine accesses a
non-existent \ac{PE}, then the \openshmem library may handle this
situation in an implementation-defined way.  For example, the library may report
an error message saying that the \ac{PE} accessed is outside the range of
accessible \acp{PE}, or may exit without a warning.\tabularnewline
\hline
Use of non-symmetric variables & Some routines require remotely accessible
variables to perform their function.  For example, a \PUT{} to a non-symmetric variable may
be trapped where possible and the library may abort the program.  Another
implementation may choose to continue execution with or without a warning.
\tabularnewline
\hline
Non-symmetric allocation of symmetric memory & The symmetric memory management routines are
collectives. For example, all \acp{PE} in the program must call
\FUNC{shmem\_malloc} with the same \VAR{size} argument.  Program behavior after a
mismatched \FUNC{shmem\_malloc} call is undefined.\tabularnewline
\hline
Use of null pointers with non-zero \VAR{len} specified & In any \openshmem routine
that takes a pointer and \VAR{len} describing the number of elements in that
pointer, a null pointer may not be given unless the corresponding \VAR{len} is also
specified as zero. Otherwise, the resulting behavior is undefined.
The following cases summarize this behavior:
\begin{itemize}
    \item \VAR{len} is 0, pointer is null: supported.
    \item \VAR{len} is not 0, pointer is null: undefined behavior.
    \item \VAR{len} is 0, pointer is non-null: supported.
    \item \VAR{len} is not 0, pointer is non-null: supported.
\end{itemize}
\tabularnewline
\hline
\end{longtable}




\chapter{History of OpenSHMEM}\label{sec:openshmem_history}

SHMEM has a long history as a parallel-programming model and has been
extensively used on a number of products since 1993, including the Cray T3D,
Cray X1E, Cray XT3 and XT4, \ac{SGI} Origin, \ac{SGI} Altix, Quadrics-based
clusters, and InfiniBand-based clusters.

\begin{itemize}
\item SHMEM Timeline
  \begin{itemize}
  \item Cray SHMEM
    \begin{itemize}
    \item SHMEM first introduced by Cray Research, Inc.\ in 1993 for Cray T3D
    \item Cray was acquired by \ac{SGI} in 1996
    \item Cray was acquired by Tera in 2000 (MTA)
    \item Platforms: Cray T3D, T3E, C90, J90, SV1, SV2, X1, X2, XE, XMT, XT
    \end{itemize}
  \item \ac{SGI} SHMEM
    \begin{itemize}
    \item \ac{SGI} acquired Cray Research, Inc.\ and SHMEM was integrated into
      \ac{SGI}'s Message Passing Toolkit (MPT)
    \item \ac{SGI} currently owns the rights to SHMEM and \openshmem
    \item Platforms: Origin, Altix 4700, Altix XE, ICE, UV
    \item \ac{SGI} was acquired by Rackable Systems in 2009
    \item \ac{SGI} and \ac{OSSS} signed a
      SHMEM trademark licensing agreement in 2010
    \item \ac{HPE} acquired {SGI} in 2016
    \end{itemize}
  \end{itemize}
\end{itemize}

A listing of \openshmem implementations can be found on
\url{http://www.openshmem.org/}.








\chapter{OpenSHMEM Specification and Deprecated API}\label{sec:dep_api}

\section{Overview}\label{subsec:dep_overview}
\TableIndex{Deprecated API}
For the \openshmem Specification, deprecation is the process of identifying
API that is supported but no longer recommended for use by users.
The deprecated API \textbf{must} be supported until clearly
indicated as otherwise by the Specification.
This chapter records the API or functionality that have been deprecated, the
version of the \openshmem Specification that effected the deprecation, and the
most recent version of the \openshmem Specification in which the feature was
supported before removal.

\begin{center}
\scriptsize
\begin{longtable}{|l|c|c|l|}
    \hline
    \textbf{Deprecated API}
    & \textbf{Deprecated Since}
    & \textbf{Last Version Supported}
    & \textbf{Replaced By} \\
    \hline
    \endhead
    Header Directory: \hyperref[subsec:dep_rationale:mpp]{\HEADER{mpp}} & 1.1 & Current & (none) \\ \hline
    \CorCpp: \hyperref[subsec:start_pes]{\FuncRef{start\_pes}} & 1.2 & Current & \hyperref[subsec:shmem_init]{\FUNC{shmem\_init}} \\ \hline
    \Fortran: \hyperref[subsec:start_pes]{\FuncRef{START\_PES}} & 1.2 & Current & \hyperref[subsec:shmem_init]{\FUNC{SHMEM\_INIT}} \\ \hline
    \hyperref[subsec:start_pes]{Implicit finalization} & 1.2 & Current & \hyperref[subsec:shmem_finalize]{\FUNC{shmem\_finalize}} \\ \hline
    \CorCpp: \FuncRef{\_my\_pe} & 1.2 & Current & \hyperref[subsec:shmem_my_pe]{\FUNC{shmem\_my\_pe}} \\ \hline
    \CorCpp: \FuncRef{\_num\_pes} & 1.2 & Current & \hyperref[subsec:shmem_n_pes]{\FUNC{shmem\_n\_pes}} \\ \hline
    \Fortran: \FuncRef{MY\_PE} & 1.2 & Current & \hyperref[subsec:shmem_my_pe]{\FUNC{SHMEM\_MY\_PE}} \\ \hline
    \Fortran: \FuncRef{NUM\_PES} & 1.2 & Current & \hyperref[subsec:shmem_n_pes]{\FUNC{SHMEM\_N\_PES}} \\ \hline
    \CorCpp: \FuncRef{shmalloc} & 1.2 & Current & \hyperref[subsec:shfree]{\FUNC{shmem\_malloc}} \\ \hline
    \CorCpp: \FuncRef{shfree} & 1.2 & Current & \hyperref[subsec:shfree]{\FUNC{shmem\_free}} \\ \hline
    \CorCpp: \FuncRef{shrealloc} & 1.2 & Current & \hyperref[subsec:shfree]{\FUNC{shmem\_realloc}} \\ \hline
    \CorCpp: \FuncRef{shmemalign} & 1.2 & Current & \hyperref[subsec:shfree]{\FUNC{shmem\_align}} \\ \hline
    \Fortran: \FuncRef{SHMEM\_PUT} & 1.2 & Current & \hyperref[subsec:shmem_put]{\FUNC{SHMEM\_PUT8} or \FUNC{SHMEM\_PUT64}} \\ \hline
    \minitab{\CorCpp: \hyperref[subsec:shmem_cache]{\FuncRef{shmem\_clear\_cache\_inv}}
        \\ \Fortran: \hyperref[subsec:shmem_cache]{\FuncRef{SHMEM\_CLEAR\_CACHE\_INV}}}
        & 1.3 & Current & (none) \\ \hline
    \CorCpp: \hyperref[subsec:shmem_cache]{\FuncRef{shmem\_clear\_cache\_line\_inv}} & 1.3 & Current & (none) \\ \hline
    \minitab{\CorCpp: \hyperref[subsec:shmem_cache]{\FuncRef{shmem\_set\_cache\_inv}}
        \\ \Fortran: \hyperref[subsec:shmem_cache]{\FuncRef{SHMEM\_SET\_CACHE\_INV}}}
        & 1.3 & Current & (none) \\ \hline
    \minitab{\CorCpp: \hyperref[subsec:shmem_cache]{\FuncRef{shmem\_set\_cache\_line\_inv}}
        \\ \Fortran: \hyperref[subsec:shmem_cache]{\FuncRef{SHMEM\_SET\_CACHE\_LINE\_INV}}}
        & 1.3 & Current & (none) \\ \hline
    \minitab{\CorCpp: \hyperref[subsec:shmem_cache]{\FuncRef{shmem\_udcflush}}
        \\ \Fortran: \hyperref[subsec:shmem_cache]{\FuncRef{SHMEM\_UDCFLUSH}}}
        & 1.3 & Current & (none) \\ \hline
    \minitab{\CorCpp: \hyperref[subsec:shmem_cache]{\FuncRef{shmem\_udcflush\_line}}
        \\ \Fortran: \hyperref[subsec:shmem_cache]{\FuncRef{SHMEM\_UDCFLUSH\_LINE}}}
        & 1.3 & Current & (none) \\ \hline
    \LibConstRef{\_SHMEM\_SYNC\_VALUE}         & 1.3 & Current & \hyperref[subsec:library_constants]{\CONST{SHMEM\_SYNC\_VALUE}} \\ \hline
    \LibConstRef{\_SHMEM\_BARRIER\_SYNC\_SIZE} & 1.3 & Current & \hyperref[subsec:library_constants]{\CONST{SHMEM\_BARRIER\_SYNC\_SIZE}} \\ \hline
    \LibConstRef{\_SHMEM\_BCAST\_SYNC\_SIZE}   & 1.3 & Current & \hyperref[subsec:library_constants]{\CONST{SHMEM\_BCAST\_SYNC\_SIZE}} \\ \hline
    \LibConstRef{\_SHMEM\_COLLECT\_SYNC\_SIZE} & 1.3 & Current & \hyperref[subsec:library_constants]{\CONST{SHMEM\_COLLECT\_SYNC\_SIZE}} \\ \hline
    \LibConstRef{\_SHMEM\_REDUCE\_SYNC\_SIZE}  & 1.3 & Current & \hyperref[subsec:library_constants]{\CONST{SHMEM\_REDUCE\_SYNC\_SIZE}} \\ \hline
    \LibConstRef{\_SHMEM\_REDUCE\_MIN\_WRKDATA\_SIZE} & 1.3 & Current & \hyperref[subsec:library_constants]{\CONST{SHMEM\_REDUCE\_MIN\_WRKDATA\_SIZE}} \\ \hline
    \LibConstRef{\_SHMEM\_MAJOR\_VERSION} & 1.3 & Current & \hyperref[subsec:library_constants]{\CONST{SHMEM\_MAJOR\_VERSION}} \\ \hline
    \LibConstRef{\_SHMEM\_MINOR\_VERSION} & 1.3 & Current & \hyperref[subsec:library_constants]{\CONST{SHMEM\_MINOR\_VERSION}} \\ \hline
    \LibConstRef{\_SHMEM\_MAX\_NAME\_LEN} & 1.3 & Current & \hyperref[subsec:library_constants]{\CONST{SHMEM\_MAX\_NAME\_LEN}} \\ \hline
    \LibConstRef{\_SHMEM\_VENDOR\_STRING} & 1.3 & Current & \hyperref[subsec:library_constants]{\CONST{SHMEM\_VENDOR\_STRING}} \\ \hline
    \LibConstRef{\_SHMEM\_CMP\_EQ} & 1.3 & Current & \hyperref[subsec:library_constants]{\CONST{SHMEM\_CMP\_EQ}} \\ \hline
    \LibConstRef{\_SHMEM\_CMP\_NE} & 1.3 & Current & \hyperref[subsec:library_constants]{\CONST{SHMEM\_CMP\_NE}} \\ \hline
    \LibConstRef{\_SHMEM\_CMP\_LT} & 1.3 & Current & \hyperref[subsec:library_constants]{\CONST{SHMEM\_CMP\_LT}} \\ \hline
    \LibConstRef{\_SHMEM\_CMP\_LE} & 1.3 & Current & \hyperref[subsec:library_constants]{\CONST{SHMEM\_CMP\_LE}} \\ \hline
    \LibConstRef{\_SHMEM\_CMP\_GT} & 1.3 & Current & \hyperref[subsec:library_constants]{\CONST{SHMEM\_CMP\_GT}} \\ \hline
    \LibConstRef{\_SHMEM\_CMP\_GE} & 1.3 & Current & \hyperref[subsec:library_constants]{\CONST{SHMEM\_CMP\_GE}} \\ \hline
    \EnvVarRef{SMA\_VERSION}         & 1.4 & Current & \hyperref[subsec:environment_variables]{\ENVVAR{SHMEM\_VERSION}} \\ \hline
    \EnvVarRef{SMA\_INFO}            & 1.4 & Current & \hyperref[subsec:environment_variables]{\ENVVAR{SHMEM\_INFO}} \\ \hline
    \EnvVarRef{SMA\_SYMMETRIC\_SIZE} & 1.4 & Current & \hyperref[subsec:environment_variables]{\ENVVAR{SHMEM\_SYMMETRIC\_SIZE}} \\ \hline
    \EnvVarRef{SMA\_DEBUG}           & 1.4 & Current & \hyperref[subsec:environment_variables]{\ENVVAR{SHMEM\_DEBUG}} \\ \hline
    \minitab{\CorCpp: \FuncRef{shmem\_wait}
        \\ \CorCpp: \FuncRef{shmem\_\FuncParam{TYPENAME}\_wait}}
        & 1.4 & Current & See \textbf{Notes} for \hyperref[subsec:shmem_wait_until]{\FUNC{shmem\_wait\_until}} \\ \hline
    \CorCpp: \FuncRef{shmem\_wait\_until} & 1.4 & Current
        & \Cstd[11]: \hyperref[subsec:shmem_wait_until]{\FUNC{shmem\_wait\_until}}, \CorCpp: \hyperref[subsec:shmem_wait_until]{\FUNC{shmem\_long\_wait\_until}} \\ \hline
    \minitab{\Cstd[11]: \FuncRef{shmem\_fetch}
        \\ \CorCpp: \FuncRef{shmem\_\FuncParam{TYPENAME}\_fetch}}
        & 1.4 & Current & \hyperref[subsec:shmem_atomic_fetch]{\FUNC{shmem\_atomic\_fetch}} \\ \hline
    \minitab{\Cstd[11]: \FuncRef{shmem\_set}
        \\ \CorCpp: \FuncRef{shmem\_\FuncParam{TYPENAME}\_set}}
        & 1.4 & Current & \hyperref[subsec:shmem_atomic_set]{\FUNC{shmem\_atomic\_set}} \\ \hline
    \minitab{\Cstd[11]: \FuncRef{shmem\_cswap}
        \\ \CorCpp: \FuncRef{shmem\_\FuncParam{TYPENAME}\_cswap}}
        & 1.4 & Current & \hyperref[subsec:shmem_atomic_compare_swap]{\FUNC{shmem\_atomic\_compare\_swap}} \\ \hline
    \minitab{\Cstd[11]: \FuncRef{shmem\_swap}
        \\ \CorCpp: \FuncRef{shmem\_\FuncParam{TYPENAME}\_swap}}
        & 1.4 & Current & \hyperref[subsec:shmem_atomic_swap]{\FUNC{shmem\_atomic\_swap}} \\ \hline
    \minitab{\Cstd[11]: \FuncRef{shmem\_finc}
        \\ \CorCpp: \FuncRef{shmem\_\FuncParam{TYPENAME}\_finc}}
        & 1.4 & Current & \hyperref[subsec:shmem_atomic_fetch_inc]{\FUNC{shmem\_atomic\_fetch\_inc}} \\ \hline
    \minitab{\Cstd[11]: \FuncRef{shmem\_inc}
        \\ \CorCpp: \FuncRef{shmem\_\FuncParam{TYPENAME}\_inc}}
        & 1.4 & Current & \hyperref[subsec:shmem_atomic_inc]{\FUNC{shmem\_atomic\_inc}} \\ \hline
    \minitab{\Cstd[11]: \FuncRef{shmem\_fadd}
        \\ \CorCpp: \FuncRef{shmem\_\FuncParam{TYPENAME}\_fadd}}
        & 1.4 & Current & \hyperref[subsec:shmem_atomic_fetch_add]{\FUNC{shmem\_atomic\_fetch\_add}} \\ \hline
    \minitab{\Cstd[11]: \FuncRef{shmem\_add}
        \\ \CorCpp: \FuncRef{shmem\_\FuncParam{TYPENAME}\_add}}
        & 1.4 & Current & \hyperref[subsec:shmem_atomic_add]{\FUNC{shmem\_atomic\_add}} \\ \hline
    Entire \Fortran API & 1.4 & Current & (none) \\ \hline
    \end{longtable}
\end{center}

\section{Deprecation Rationale}\label{subsec:dep_rationale}

\subsection{Header Directory: \HEADER{mpp}}
\label{subsec:dep_rationale:mpp}
In addition to the default system header paths, \openshmem implementations
must provide all \openshmem-specified header files from the \HEADER{mpp}
header directory such that these headers can be referenced in \CorCpp as
\begin{lstlisting}[language=]
#include <mpp/shmem.h>
#include <mpp/shmemx.h>
\end{lstlisting}
and in \Fortran as
\begin{lstlisting}[language=]
include 'mpp/shmem.fh'
include 'mpp/shmemx.fh'
\end{lstlisting}
for backwards compatibility with \ac{SGI} SHMEM.

\subsection{\CorCpp: \FUNC{start\_pes}}
The \CorCpp routine \FUNC{start\_pes} includes an unnecessary initialization
argument that is remnant of historical \emph{SHMEM} implementations and no
longer reflects the requirements of modern \openshmem implementations.
Furthermore, the naming of \FUNC{start\_pes} does not include the standardized
\shmemprefixLC{} naming prefix. This routine has been deprecated and
\openshmem users are encouraged to use \FUNC{shmem\_init} instead.

\subsection{Implicit Finalization}
Implicit finalization was deprecated and replaced with explicit finalization using the
\FUNC{shmem\_finalize} routine.  Explicit finalization improves portability and
also improves interoperability with profiling and debugging tools.

\subsection{\CorCpp: \FUNC{\_my\_pe}, \FUNC{\_num\_pes}, \FUNC{shmalloc},
    \FUNC{shfree}, \FUNC{shrealloc}, \FUNC{shmemalign}}
The \CorCpp routines \FUNC{\_my\_pe}, \FUNC{\_num\_pes}, \FUNC{shmalloc},
\FUNC{shfree}, \FUNC{shrealloc}, and \FUNC{shmemalign} were deprecated in order
to normalize the \openshmem \ac{API} to use \shmemprefixLC{} as the standard
prefix for all routines.

\subsection{\textit{Fortran}: \FUNC{START\_PES}, \FUNC{MY\_PE}, \FUNC{NUM\_PES}} %% WARNING: Issue #66.
The \Fortran routines \FUNC{START\_PES}, \FUNC{MY\_PE}, and \FUNC{NUM\_PES}
were deprecated in order to minimize the API differences from the deprecation
of \CorCpp routines \FUNC{start\_pes}, \FUNC{\_my\_pe}, and \FUNC{\_num\_pes}.

\subsection{\textit{Fortran}: \FUNC{SHMEM\_PUT}} %% WARNING: Issue #66.
The \Fortran routine \FUNC{SHMEM\_PUT} is defined only for the \Fortran
\ac{API} and is semantically identical to \Fortran routines
\FUNC{SHMEM\_PUT8} and \FUNC{SHMEM\_PUT64}.  Since \FUNC{SHMEM\_PUT8} and
\FUNC{SHMEM\_PUT64} have defined equivalents in the \CorCpp interface,
\FUNC{SHMEM\_PUT} is ambiguous and has been deprecated.

\subsection{SHMEM\_CACHE}
The \FUNC{SHMEM\_CACHE} \ac{API}
\begin{center}
\begin{tabular}{ll}
    \CorCpp: & \Fortran: \\
    \FUNC{shmem\_clear\_cache\_inv}     & \FUNC{SHMEM\_CLEAR\_CACHE\_INV} \\
    \FUNC{shmem\_set\_cache\_inv}       & \FUNC{SHMEM\_SET\_CACHE\_INV} \\
    \FUNC{shmem\_set\_cache\_line\_inv} & \FUNC{SHMEM\_SET\_CACHE\_LINE\_INV} \\
    \FUNC{shmem\_udcflush}              & \FUNC{SHMEM\_UDCFLUSH} \\
    \FUNC{shmem\_udcflush\_line}        & \FUNC{SHMEM\_UDCFLUSH\_LINE} \\
    \FUNC{shmem\_clear\_cache\_line\_inv} \\
\end{tabular}
\end{center}
was originally implemented for systems with cache-management instructions.
This API has largely gone unused on cache-coherent system architectures.
\FUNC{SHMEM\_CACHE} has been deprecated.

\subsection{\CONST{\_SHMEM\_*} Library Constants}
The library constants
\begin{center}
\begin{tabular}{ll}
    \CONST{\_SHMEM\_SYNC\_VALUE}         & \CONST{\_SHMEM\_MAX\_NAME\_LEN} \\
    \CONST{\_SHMEM\_BARRIER\_SYNC\_SIZE} & \CONST{\_SHMEM\_VENDOR\_STRING} \\
    \CONST{\_SHMEM\_BCAST\_SYNC\_SIZE}   & \CONST{\_SHMEM\_CMP\_EQ} \\
    \CONST{\_SHMEM\_COLLECT\_SYNC\_SIZE} & \CONST{\_SHMEM\_CMP\_NE} \\
    \CONST{\_SHMEM\_REDUCE\_SYNC\_SIZE}  & \CONST{\_SHMEM\_CMP\_LT} \\
    \CONST{\_SHMEM\_REDUCE\_MIN\_WRKDATA\_SIZE} & \CONST{\_SHMEM\_CMP\_LE} \\
    \CONST{\_SHMEM\_MAJOR\_VERSION}      & \CONST{\_SHMEM\_CMP\_GT} \\
    \CONST{\_SHMEM\_MINOR\_VERSION}      & \CONST{\_SHMEM\_CMP\_GE} \\
\end{tabular}
\end{center}
do not adhere to the \Cstd standard's reserved identifiers and the \Cpp
standard's reserved names.  These constants were deprecated and replaced
with corresponding constants of prefix \shmemprefix{} that adhere to \CorCpp{}
and \Fortran naming conventions.

\subsection{\ENVVAR{SMA\_*} Environment Variables}\label{subsec:deprecate-sma-env}
The environment variables \ENVVAR{SMA\_VERSION}, \ENVVAR{SMA\_INFO},
\ENVVAR{SMA\_SYMMETRIC\_SIZE}, and \ENVVAR{SMA\_DEBUG}
were deprecated in order to normalize the \openshmem \ac{API} to use
\shmemprefix{} as the standard prefix for all environment variables.

\subsection{\CorCpp: \FUNC{shmem\_wait}}
The \CorCpp interface for \FUNC{shmem\_wait} and \FUNC{shmem\_\FuncParam{TYPENAME}\_wait}
was identified as unintuitive with respect to
the comparison operation it performed.  As \FUNC{shmem\_wait} can be trivially
replaced by \FUNC{shmem\_wait\_until} where \VAR{cmp} is
\CONST{SHMEM\_CMP\_NE}, the \FUNC{shmem\_wait} interface was deprecated in
favor of \FUNC{shmem\_wait\_until}, which makes the comparison operation
explicit and better communicates the developer's intent.

\subsection{\CorCpp: \FUNC{shmem\_wait\_until}}
The \CTYPE{long}-typed \CorCpp routine \FUNC{shmem\_wait\_until} was deprecated
in favor of the \Cstd[11] type-generic interface of the same name or the
explicitly typed \CorCpp routine \FUNC{shmem\_long\_wait\_until}.

\subsection{\textit{C11} and \CorCpp: \FUNC{shmem\_fetch}, \FUNC{shmem\_set}, %% Issue #66.
    \FUNC{shmem\_cswap}, \FUNC{shmem\_swap}, \FUNC{shmem\_finc},
    \FUNC{shmem\_inc}, \FUNC{shmem\_fadd}, \FUNC{shmem\_add}}
The \Cstd[11] and \CorCpp interfaces for
\begin{center}
\begin{tabular}{ll}
    \Cstd[11]: & \CorCpp: \\
    \FUNC{shmem\_fetch} & \FUNC{shmem\_\FuncParam{TYPENAME}\_fetch} \\
    \FUNC{shmem\_set}   & \FUNC{shmem\_\FuncParam{TYPENAME}\_set}   \\
    \FUNC{shmem\_cswap} & \FUNC{shmem\_\FuncParam{TYPENAME}\_cswap} \\
    \FUNC{shmem\_swap}  & \FUNC{shmem\_\FuncParam{TYPENAME}\_swap}  \\
    \FUNC{shmem\_finc}  & \FUNC{shmem\_\FuncParam{TYPENAME}\_finc}  \\
    \FUNC{shmem\_inc}   & \FUNC{shmem\_\FuncParam{TYPENAME}\_inc}   \\
    \FUNC{shmem\_fadd}  & \FUNC{shmem\_\FuncParam{TYPENAME}\_fadd}  \\
    \FUNC{shmem\_add}   & \FUNC{shmem\_\FuncParam{TYPENAME}\_add}   \\
\end{tabular}
\end{center}
were deprecated and replaced with
similarly named interfaces within the \FUNC{shmem\_atomic\_*} namespace
in order to more clearly identify these calls as performing atomic operations.
In addition, the abbreviated names ``cswap'', ``finc'', and ``fadd'' were
expanded for clarity to ``compare\_swap'', ``fetch\_inc'', and ``fetch\_add''.

\subsection{\textit{Fortran} API}\label{subsec:deprecate-fortran} %% WARNING: Issue #66.
The entire \openshmem \Fortran API was deprecated because of a general lack of
use and a lack of conformance with legacy \Fortran standards. In lieu of an
extensive update of the \Fortran API, \Fortran users are encouraged to
leverage the \openshmem Specification's \Cstd API through the
\Fortran--\Cstd interoperability initially standardized by \Fortran[2003]%
\footnote{Formally, \Fortran[2003] is known as ISO/IEC~1539-1:2004(E).}.





\chapter{Changes to this Document}\label{sec:changelog}

\section{Version 1.5}
Major changes in \openshmem[1.5] include \dots

The following list describes the specific changes in \openshmem[1.5]:
\begin{itemize}
%
\item Added support for nonblocking put-with-signal functions.
\\ See Section \ref{subsec:shmem_put_signal_nbi}.
%
\item Specified the validity of communication contexts, added the constant
  \CONST{SHMEM\_CTX\_INVALID}, and clarified the behavior of
  \FUNC{shmem\_ctx\_*} routines on invalid contexts.
  \\ See Section~\ref{sec:ctx}.
%
\item Clarified \ac{PE} active set requirements.
    \\See Section~\ref{subsec:coll}.
%
\item Clarified that when the \VAR{size} argument is zero, symmetric heap
    allocation routines perform no action and return a null pointer; that
    symmetric heap management routines that perform no action do not perform a
    barrier; and that the \VAR{alignment} argument to \FUNC{shmem\_align} must
    be power of two multiple of \CONST{sizeof(void*)}.
    \\See Section~\ref{subsec:shfree}.
%
\item Clarified that the \openshmem lock API provides a non-reentrant mutex and
    that \FUNC{shmem\_clear\_lock} performs a quiet operation on the default
    context.
    \\See Section~\ref{subsec:shmem_lock}
%
\item Clarified the atomicity guarantees of the \openshmem memory model.
    \\See Section~\ref{subsec:amo_guarantees}.
%
\end{itemize}

\section{Version 1.4}
Major changes in \openshmem[1.4] include
multithreading support,
\emph{contexts} for communication management,
\FUNC{shmem\_sync},
\FUNC{shmem\_calloc},
expanded type support,
a new namespace for atomic operations,
atomic bitwise operations,
\FUNC{shmem\_test} for nonblocking point-to-point synchronization,
and \Cstd[11] type-generic interfaces for point-to-point synchronization.

The following list describes the specific changes in \openshmem[1.4]:
\begin{itemize}
%
\item New communication management API, including \FUNC{shmem\_ctx\_create};
    \FUNC{shmem\_ctx\_destroy}; and additional RMA, AMO, and memory ordering
    routines that accept \CTYPE{shmem\_ctx\_t} arguments.
\\See Section \ref{sec:ctx}.
%
\item New API \FUNC{shmem\_sync\_all} and \FUNC{shmem\_sync} to provide \ac{PE}
    synchronization without completing pending communication operations.
    \\See Sections \ref{subsec:shmem_sync_all} and \ref{subsec:shmem_sync}.
%
\item Clarified that the \openshmem extensions header files are required, even when empty.
\\See Section~\ref{subsec:bindings}.
%
\item Clarified that the \FUNC{SHMEM\_GET64} and \FUNC{SHMEM\_GET64\_NBI}
    routines are included in the \Fortran language bindings.\\
    See Sections \ref{subsec:shmem_get} and \ref{subsec:shmem_get_nbi}.
%
\item Clarified that \FUNC{shmem\_init} must be matched with a call to
    \FUNC{shmem\_finalize}.
\\See Sections \ref{subsec:shmem_init} and \ref{subsec:shmem_finalize}.
%
\item Added the \CONST{SHMEM\_SYNC\_SIZE} constant.
\\See Section \ref{subsec:library_constants}.
%
\item Added type-generic interfaces for \FUNC{shmem\_wait\_until}.
\\ See Section \ref{subsec:shmem_wait_until}.
%
\item Removed the \VAR{volatile} qualifiers from the \VAR{ivar} arguments to
\FUNC{shmem\_wait} routines and the \VAR{lock} arguments in the lock API.
\emph{Rationale: Volatile qualifiers were added to several API routines in
\openshmem[1.3]; however, they were later found to be unnecessary.}
\\ See Sections \ref{subsec:shmem_wait_until} and \ref{subsec:shmem_lock}.
%
\item Deprecated the \VAR{SMA\_}* environment variables and added equivalent
\VAR{SHMEM\_}* environment variables.
\\ See Section \ref{subsec:environment_variables}.
%
\item Added the \Cstd[11] \CTYPE{\_Noreturn} function specifier to
\FUNC{shmem\_global\_exit}.
\\ See Section \ref{subsec:shmem_global_exit}.
%
\item Clarified ordering semantics of memory ordering, point-to-point synchronization, and collective
synchronization routines.
%
\item Clarified deprecation overview and added deprecation rationale in Annex F.
\\See Section \ref{sec:dep_api}.
%
\item Deprecated header directory \HEADER{mpp}.
\\See Section \ref{sec:dep_api}.
%
\item Deprecated the \FUNC{shmem\_wait} functions and the \CTYPE{long}-typed \CorCpp \FUNC{shmem\_wait\_until} function.
\\ See Section \ref{subsec:p2p_intro}.
%
\item Added the \FUNC{shmem\_test} functions.
\\ See Section \ref{subsec:p2p_intro}.
%
\item Added the \FUNC{shmem\_calloc} function.
\\ See Section \ref{subsec:shmem_calloc}.
%
\item Introduced the thread safe semantics that define the interaction between
    \openshmem routines and user threads.
\\See Section \ref{subsec:thread_support}.
%
\item Added the new routine \FUNC{shmem\_init\_thread} to initialize the
    \openshmem library with one of the defined thread levels.
\\See Section \ref{subsec:shmem_init_thread}.
%
\item Added the new routine \FUNC{shmem\_query\_thread} to query the thread
    level provided by the \openshmem implementation.
\\See Section \ref{subsec:shmem_query_thread}.
%
\item Clarified the semantics of \FUNC{shmem\_quiet} for a multithreaded
    \openshmem \ac{PE}.
\\See Section \ref{subsec:shmem_quiet}
%
\item Revised the description of \FUNC{shmem\_barrier\_all} for a multithreaded
    \openshmem \ac{PE}.
\\See Section \ref{subsec:shmem_barrier_all}
%
\item Revised the description of \FUNC{shmem\_wait} for a multithreaded
    \openshmem \ac{PE}.
\\See Section \ref{subsec:shmem_wait_until}
%
\item Clarified description for \CONST{SHMEM\_VENDOR\_STRING}.
\\See Section \ref{subsec:library_constants}.
%
\item Clarified description for \CONST{SHMEM\_MAX\_NAME\_LEN}.
\\See Section \ref{subsec:library_constants}.
%
\item Clarified API description for \FUNC{shmem\_info\_get\_name}.
\\See Section \ref{subsec:shmem_info_get_name}.
%
\item Expanded the type support for RMA, AMO, and point-to-point
    synchronization operations.
\\ See Tables \ref{stdrmatypes}, \ref{stdamotypes}, \ref{extamotypes}, and
    \ref{p2psynctypes}
%
\item Renamed AMO operations to use \FUNC{shmem\_atomic\_*} prefix and
      deprecated old AMO routines.
\\ See Section \ref{sec:amo}.
%
\item Added fetching and non-fetching bitwise AND, OR, and XOR atomic
      operations.
\\ See Section \ref{sec:amo}.
%
\item Deprecated the entire \Fortran API.
%
\item Replaced the \CTYPE{complex} macro in complex-typed reductions with the
      \Cstd[99] (and later) type specifier \CTYPE{\_Complex} to remove an
      implicit dependence on \HEADER{complex.h}.
\\ See Section \ref{subsec:shmem_reductions}.
%
\item Clarified that complex-typed reductions in C are optionally supported.
\\ See Section \ref{subsec:shmem_reductions}.
%
\end{itemize}




\section{Version 1.3}
Major changes in \openshmem[1.3] include the addition of
nonblocking \ac{RMA} operations,
atomic \PUT{} and \GET{} operations,
all-to-all collectives,
and \Cstd[11] type-generic interfaces for \ac{RMA} and \ac{AMO} operations.

The following list describes the specific changes in \openshmem[1.3]:
\begin{itemize}
%
\item Clarified implementation of \acp{PE} as threads.
%
\item Added \CTYPE{const} to every read-only pointer argument.
%
\item Clarified definition of \OPR{Fence}.
\\See Section \ref{subsec:programming_model}.
%
\item Clarified implementation of symmetric memory allocation.
\\See Section \ref{subsec:memory_model}.
%
\item Restricted atomic operation guarantees to other atomic operations with the same datatype.
\\See Section \ref{subsec:amo_guarantees}.
%
\item Deprecation of all constants that start with \CONST{\_SHMEM\_*}.
\\See Section \ref{subsec:library_constants}.
%
\item Added a type-generic interface to \openshmem \ac{RMA} and \ac{AMO}
	operations based on \Cstd[11] Generics.
\\See Sections \ref{sec:rma}, \ref{sec:rma_nbi} and \ref{sec:amo}.
%
\item New nonblocking variants of remote memory access, \FUNC{SHMEM\_PUT\_NBI}
	and \FUNC{SHMEM\_GET\_NBI}.
\\See Sections \ref{subsec:shmem_put_nbi} and \ref{subsec:shmem_get_nbi}.
%
\item New atomic elemental read and write operations, \FUNC{SHMEM\_FETCH} and
	\FUNC{SHMEM\_SET}.
\\See Sections \ref{subsec:shmem_atomic_fetch} and \ref{subsec:shmem_atomic_set}
%
\item New alltoall data exchange operations, \FUNC{SHMEM\_ALLTOALL}
	and \FUNC{SHMEM\_ALLTOALLS}.
\\See Sections \ref{subsec:shmem_alltoall} and \ref{subsec:shmem_alltoalls}.
%
\item Added \CTYPE{volatile} to remotely accessible pointer argument in
	\FUNC{SHMEM\_WAIT} and \FUNC{SHMEM\_LOCK}.
\\See Sections \ref{subsec:shmem_wait_until} and \ref{subsec:shmem_lock}.
%
\item Deprecation of \FUNC{SHMEM\_CACHE}.
\\See Section \ref{subsec:shmem_cache}.
%
\end{itemize}




\section{Version 1.2}
Major changes in \openshmem[1.2] include
a new initialization routine (\FUNC{shmem\_init}),
improvements to the execution model with an explicit
library-finalization routine (\FUNC{shmem\_finalize}),
an early-exit routine (\FUNC{shmem\_global\_exit}),
namespace standardization,
and clarifications to several API descriptions.

The following list describes the specific changes in \openshmem[1.2]:
\begin{itemize}
%
\item Added specification of \VAR{pSync} initialization for all routines that use it.
%
\item Replaced all placeholder variable names \VAR{target} with \VAR{dest} to
      avoid confusion with \Fortran's \KEYWORD{target} keyword.
%
\item New Execution Model for exiting/finishing \openshmem programs.
\\See Section  \ref{subsec:execution_model}.
%
\item New library constants to support API that query version and name information.
\\See Section \ref{subsec:library_constants}.
%
\item New API \FUNC{shmem\_init} to provide mechanism to start an \openshmem
      program and replace deprecated \FUNC{start\_pes}.
\\See Section \ref{subsec:shmem_init}.
%
\item Deprecation of \FUNC{\_my\_pe} and \FUNC{\_num\_pes} routines.
\\See Sections \ref{subsec:shmem_my_pe} and \ref{subsec:shmem_n_pes}.
%
\item New API \FUNC{shmem\_finalize} to provide collective mechanism to cleanly
      exit an \openshmem program and release resources.
\\See Section \ref{subsec:shmem_finalize}.
%
\item New API \FUNC{shmem\_global\_exit} to provide mechanism to exit an
    \openshmem program.
\\See Section \ref{subsec:shmem_global_exit}.
%
\item Clarification related to the address of the referenced object in
    \FUNC{shmem\_ptr}.
\\See Section \ref{subsec:shmem_ptr}.
%
\item New API to query the version and name information.
\\See Section \ref{subsec:shmem_info_get_version} and \ref{subsec:shmem_info_get_name}.
%
\item \openshmem library API normalization. All \Cstd symmetric memory management
      API begins with  \FUNC{shmem\_}.
\\See Section \ref{subsec:shfree}.
%
\item Notes and clarifications added to \FUNC{shmem\_malloc}.
\\See Section \ref{subsec:shfree}.
%
\item Deprecation of \Fortran API routine \FUNC{SHMEM\_PUT}.
\\See Section \ref{subsec:shmem_put}.
%
\item Clarification related to \FUNC{shmem\_wait}.
\\See Section \ref{subsec:shmem_wait_until}.
%
\item Undefined behavior for null pointers without zero counts added.
\\See Annex \ref{sec:undefined}
%
\item Addition of new Annex for clearly specifying deprecated API and its
      support across versions of the \openshmem Specification.
\\See Annex \ref{sec:dep_api}.
%
\end{itemize}




\section{Version 1.1}
Major changes from \openshmem[1.0] to \openshmem[1.1] include
the introduction of the \HEADER{shmemx.h} header file for non-standard API
extensions,
clarifications to completion semantics and API descriptions in agreement with
the \ac{SGI} SHMEM specification,
and general readabilty and usability improvements to the document structure.

The following list describes the specific changes in \openshmem[1.1]:
\begin{itemize}
%
\item Clarifications of the completion semantics of memory synchronization
      interfaces.
\\See Section \ref{subsec:memory_order}.
%
\item Clarification of the completion semantics of memory load and store
      operations in context of \FUNC{shmem\_barrier\_all} and \FUNC{shmem\_barrier}
      routines.
\\See Section \ref{subsec:shmem_barrier_all} and \ref{subsec:shmem_barrier}.
%
\item Clarification of the completion and ordering semantics of
      \FUNC{shmem\_quiet} and \FUNC{shmem\_fence}.
\\See Section \ref{subsec:shmem_quiet} and \ref{subsec:shmem_fence}.
%
\item Clarifications of the completion semantics of \ac{RMA} and \ac{AMO}
      routines.
\\See Sections \ref{sec:rma} and \ref{sec:amo}
%
\item Clarifications of the memory model and the memory alignment requirements
      for symmetric data objects.
\\See Section \ref{subsec:memory_model}.
%
\item Clarification of the execution model and the definition of a \ac{PE}.
\\See Section \ref{subsec:execution_model}
%
\item Clarifications of the semantics of \FUNC{shmem\_pe\_accessible} and
      \FUNC{shmem\_addr\_accessible}.
\\See Section \ref{subsec:shmem_pe_accessible} and \ref{subsec:shmem_addr_accessible}.
%
\item Added an annex on interoperability with \ac{MPI}.
\\See Annex \ref{sec:mpi}.
%
\item Added examples to the different interfaces.
%
\item Clarification of the naming conventions for constant in \Cstd and
      \Fortran.
\\See Section \ref{subsec:library_constants} and \ref{subsec:shmem_wait_until}.
%
\item Added \ac{API} calls: \FUNC{shmem\_char\_p}, \FUNC{shmem\_char\_g}.
\\See Sections \ref{subsec:shmem_p} and \ref{subsec:shmem_g}.
%
\item Removed \ac{API} calls: \FUNC{shmem\_char\_put},
      \FUNC{shmem\_char\_get}.
\\See Sections \ref{subsec:shmem_put} and \ref{subsec:shmem_get}.
%
\item The usage of \CTYPE{ptrdiff\_t}, \CTYPE{size\_t}, and \CTYPE{int} in the
      interface signature was made consistent with the description.
\\See Sections \ref{subsec:coll}, \ref{subsec:shmem_iput}, and \ref{subsec:shmem_iget}.
%
\item Revised \FUNC{shmem\_barrier} example.
\\See Section \ref{subsec:shmem_barrier}.
%
\item Clarification of the initial value of \VAR{pSync} work arrays for
\FUNC{shmem\_barrier}.\\ See Section \ref{subsec:shmem_barrier}.
%
\item Clarification of the expected behavior when multiple \FUNC{start\_pes}
calls are encountered.
\\See Section \ref{subsec:start_pes}.
%
\item Corrected the definition of atomic increment operation.
\\See Section \ref{subsec:shmem_atomic_inc}.
%
\item Clarification of the size of the symmetric heap and when it is set.
\\See Section \ref{subsec:shfree}.
%
\item Clarification of the integer and real sizes for \Fortran \ac{API}.
\\See Sections \ref{subsec:shmem_atomic_add},
      \ref{subsec:shmem_atomic_compare_swap},
      \ref{subsec:shmem_atomic_swap},
      \ref{subsec:shmem_atomic_fetch_inc},
      \ref{subsec:shmem_atomic_inc}, and
      \ref{subsec:shmem_atomic_fetch_add}.
%
\item Clarification of the expected behavior on program \OPR{exit}.
\\See Section \ref{subsec:execution_model}, Execution Model.
%
\item More detailed description for the progress of \openshmem operations
provided.
\\See Section \ref{subsec:progress}.
%
\item Clarification of naming convention for non-standard interfaces and their
inclusion in \HEADER{shmemx.h}.
\\See Section \ref{subsec:bindings}.
%
\item Various fixes to \openshmem code examples across the Specification to
include appropriate header files.
%
\item Removing requirement that implementations should detect size mismatch and
return error information for \FUNC{shmalloc} and ensuring consistent
language.
\\See Sections \ref{subsec:shfree} and Annex \ref{sec:undefined}.
%
\item \Fortran programming fixes for examples.\\ See Sections
\ref{subsec:shmem_reductions} and \ref{subsec:shmem_wait_until}.
%
\item Clarifications of the reuse \VAR{pSync} and \VAR{pWork} across
collectives.
\\See Sections \ref{subsec:coll}, \ref{subsec:shmem_broadcast},
      \ref{subsec:shmem_collect} and \ref{subsec:shmem_reductions}.
%
\item Name changes for UV and ICE for \ac{SGI} systems.
\\See Annex \ref{sec:openshmem_history}.
%
\end{itemize}

} %end of setlength command that was started in frontmatter.tex


\clearpage
\phantomsection
\addcontentsline{toc}{chapter}{Index}
\printindex

\end{document}

