\color{Green}
\apisummary{
    The nonblocking put-with-signal routines provide a method for copying data
    from a contiguous local data object to a data object on a specified \ac{PE}
    and subsequently setting a remote flag to signal completion.
}

\begin{apidefinition}

\begin{C11synopsis}
void @\FuncDecl{shmem\_put\_signal\_nbi}@(TYPE *dest, const TYPE *source, size_t nelems, uint64_t *restrict sig_addr, uint64_t signal, int pe);
void @\FuncDecl{shmem\_put\_signal\_nbi}@(shmem_ctx_t ctx, TYPE *dest, const TYPE *source, size_t nelems, uint64_t *restrict sig_addr, uint64_t signal, int pe);
\end{C11synopsis}
where \TYPE{} is one of the standard \ac{RMA} types specified by Table \ref{stdrmatypes}.

\begin{Csynopsis}
void @\FuncDecl{shmem\_\FuncParam{TYPENAME}\_put\_signal\_nbi}@(TYPE *dest, const TYPE *source, size_t nelems, uint64_t *restrict sig_addr, uint64_t signal, int pe);
void @\FuncDecl{shmem\_ctx\_\FuncParam{TYPENAME}\_put\_signal\_nbi}@(shmem_ctx_t ctx, TYPE *dest, const TYPE *source, size_t nelems, uint64_t *restrict sig_addr, uint64_t signal, int pe);
\end{Csynopsis}
where \TYPE{} is one of the standard \ac{RMA} types and has a corresponding \TYPENAME{} specified by Table \ref{stdrmatypes}.

\begin{CsynopsisCol}
void @\FuncDecl{shmem\_put\FuncParam{SIZE}\_signal\_nbi}@(void *dest, const void *source, size_t nelems, uint64_t *restrict sig_addr, uint64_t signal, int pe);
void @\FuncDecl{shmem\_ctx\_put\FuncParam{SIZE}\_signal\_nbi}@(shmem_ctx_t ctx, void *dest, const void *source, size_t nelems, uint64_t *restrict sig_addr, uint64_t signal, int pe);
\end{CsynopsisCol}
where \SIZE{} is one of \CONST{8, 16, 32, 64, 128}.

\begin{CsynopsisCol}
void @\FuncDecl{shmem\_putmem\_signal\_nbi}@(void *dest, const void *source, size_t nelems, uint64_t *restrict sig_addr, uint64_t signal, int pe);
void @\FuncDecl{shmem\_ctx\_putmem\_signal\_nbi}@(shmem_ctx_t ctx, void *dest, const void *source, size_t nelems, uint64_t *restrict sig_addr, uint64_t signal, int pe);
\end{CsynopsisCol}

\begin{apiarguments}
    \apiargument{IN}{ctx}{The context on which to perform the operation.
      When this argument is not provided, the operation is performed on
      \CONST{SHMEM\_CTX\_DEFAULT}.}
    \apiargument{OUT}{dest}{Data object to be updated on the remote \ac{PE}. This
    data object must be remotely accessible.}
    \apiargument{IN}{source}{Data object containing the data to be copied.}
    \apiargument{IN}{nelems}{Number of elements in the \VAR{dest} and \VAR{source}
    arrays. \VAR{nelems} must be of type \VAR{size\_t} for \Cstd.}
    \apiargument{OUT}{sig\_addr}{Data object to be updated on the remote
    \ac{PE} as the signal. This signal data object must be
    remotely accessible.}
    \apiargument{IN}{signal}{Unsigned 64-bit value that is assigned to the
    remote \VAR{sig\_addr} signal data object.}
    \apiargument{IN}{pe}{\ac{PE} number of the remote \ac{PE}.}
\end{apiarguments}

\apidescription{
    The routines return after posting the operation. The operation is considered
    complete after the subsequent call to \FUNC{shmem\_quiet}. At the completion
    of \FUNC{shmem\_quiet}, the data has been copied out of the \VAR{source}
    array on the local \ac{PE} and delivered into the \VAR{dest} array on the
    destination \ac{PE}. The delivery of \VAR{signal} flag on the remote
    \ac{PE} indicates the delivery of its corresponding \VAR{dest} data words
    into the data object on the remote \ac{PE}.
}

\apireturnvalues{
    None.
}

\apinotes{
    The \VAR{dest} and \VAR{sig\_addr} data objects must both be remotely
    accessible. The \VAR{sig\_addr} and \VAR{dest} could be of different kinds,
    for example, one could be a global/static \Cstd variable and the other could
    be allocated on the symmetric heap.

    The restrict qualifier in \VAR{sig\_addr} expects the data object to be
    distinct from \VAR{dest} and \VAR{source} data objects.

    The delivery of \VAR{signal} flag on the remote \ac{PE} indicates only the
    delivery of its corresponding \VAR{dest} data words into the data object on
    the remote \ac{PE}. Without a memory-ordering operation, there is no implied
    ordering between the delivery of the signal word of a nonblocking
    put-with-signal routine and another data transfer. For example, the delivery
    of the signal word in a sequence consisting of a put routine followed by a
    nonblocking put-with-signal routine does not imply delivery of the put
    routine's data.
}

\end{apidefinition}
\color{Black}
