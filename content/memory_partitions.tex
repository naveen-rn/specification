\emph{Symmetric Heap} is created on a single memory location determined by the
implementation or at multiple memory locations determined by the users. The
user-determined memory locations are called as the \emph{Symmetric Memory
Partitions}. Only a single symmetric heap is created at each symmetric memory
partition. Each symmetric memory partition is identified using its
\emph{Partition ID}, which is used in dynamic OpenSHMEM memory allocation
(\textit{shmem\_partition\_malloc} and \textit{shmem\_partition\_align})
routines. Apart from the Partition ID, each symmetric memory partition have
their own memory traits. Refer to Section \ref{subsec:environment_variables}
for a complete list of traits available to define the characteristics of a
memory partition. Multiple symmetric heaps can be created by defining multiple
separate memory partitions. Apart from name, type, and size attributes,
symmetric data objects stored on symmetric heaps created over a memory
partition have Partition ID as an extra attribute.