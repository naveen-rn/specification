\emph{Symmetric Heaps} can be created on a single memory location determined by
the implementation or on multiple memory locations determined by the users. The
user-determined memory locations are called as the \emph{Symmetric Memory
Partitions}. Only a single symmetric heap can be created on each symmetric
memory partition. Each symmetric memory partitions is identified using its
\emph{Partition ID}, which is used on dynamic OpenSHMEM memory allocation 
(\textit{shmem\_kind\_malloc} and \textit{shmem\_kind\_align}) routines. Apart
from the Partition ID, each symmetric memory partition have their own memory
properties. Refer to Section \ref{subsec:environment_variables} for the
complete list of properties which are available to characterize a memory
partition. Multiple symmetric heaps can be created by defining multiple
separate memory partitions. Apart from name, type and size attributes,
symmetric data objects stored on symmetric heaps created over a memory
partition have Partition ID as an extra attribute.