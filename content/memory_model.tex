\begin{figure}[h]
\includegraphics[width=0.95\textwidth]{figures/mem_model}      
\caption{\OSH Memory Model}                                   
\label{fig:mem_model}                                               
\end{figure}      
%
An \openshmem program consists of data objects that are private to each \ac{PE}
and data  objects that are remotely accessible by all \ac{PE}s. Private data
objects are stored in the local memory of each \ac{PE} and can only be accessed
by the \ac{PE} itself; these data objects cannot be accessed by other \ac{PE}s
via \openshmem routines. Private data objects follow the memory model of
\Clang{} or \Fortran. Remotely accessible objects, however, can be accessed by
remote \ac{PE}s using \openshmem routines.  Remotely accessible data objects are
called \emph{Symmetric Data Objects}.  Each symmetric data object has a
corresponding object with the same name, type, and size on all PEs where that object is
accessible via the \openshmem \ac{API}\footnote{For efficiency reasons,
the same offset (from an arbitrary memory address) for symmetric data
objects might be used on all \acp{PE}. Further discussion about symmetric heap
layout and implementation efficiency can be found in section
\ref{subsec:shfree}}.  (For the definition of what is accessible, see the
descriptions for \FUNC{shmem\_pe\_accessible} and \FUNC{shmem\_addr\_accessible}
in sections \ref{subsec:shmem_pe_accessible} and
\ref{subsec:shmem_addr_accessible}.) Symmetric data objects accessed via typed
\openshmem interfaces are required to be natural aligned based on their type
requirements and underlying architecture.  In \openshmem the following kinds of
data objects are symmetric:
%
\begin{itemize}
  \item \Fortran{} data objects in common blocks or with the  SAVE  attribute.
      These data objects must not be defined in a dynamic shared object (DSO).
  \item Global and static \Clang{} and \Cpp variables. These data objects must
      not  be defined in a DSO.
  \item \Fortran{} arrays allocated with \textit{shpalloc} 
  \item \Clang{} and \Cpp data allocated by \textit{shmem\_malloc}
\end{itemize}       

\openshmem dynamic memory allocation routines (\textit{shpalloc} and
\textit{shmem\_malloc}) allow collective allocation of \emph{Symmetric Data
Objects} on a special memory region called the \emph{Symmetric Heap}. The
Symmetric Heap is created during the execution of a program at a memory location
determined by the implementation \newtext{or on multiple user determined memory 
locations called as the \emph{Symmetric Memory Partitions}}. The Symmetric 
Heap may reside in different memory regions on different \acp{PE}. 
Figure~\ref{fig:mem_model} shows  how \openshmem implements a \ac{PGAS} model 
using remotely accessible symmetric objects and private data objects when executing 
an \openshmem program. Symmetric data objects are stored on the symmetric heap or 
in the global/static memory section of each \ac{PE}. 
