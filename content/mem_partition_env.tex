All OpenSHMEM implementations supporting the use of multiple Symmetric Memory
Partitions provide support for a special environment variable called
SHMEM\_SYMMETRIC\_PARTITION to define the properties of the user-defined memory
locations which can be used to create symmetric heaps. The properties of the
memory partitions including the size of the symmetric heap to be created are
passed as specifiers to the environment variable.

\begin{envvardefinition}
\begin{envvarname}
SHMEM_SYMMETRIC_PARTITION<ID>=SIZE=<size>[:PGSIZE=<pgsize>][:KIND=<kind>:POLICY=<policy>]
\end{envvarname}

\begin{envvararguments}
    \envvarargument{REQ}{ID}{Integer to represent the partition number}
    \envvarargument{REQ}{SIZE}{Number of bytes to allocate for symmetric heap}
    \envvarargument{OPT}{PGSIZE}{Number of bytes to specify the memory page size}
    \envvarargument{OPT}{KIND}{String constants to specify the memory kind if
    multiple memory kinds are supported by the system and identified by the
    implementation}
    \envvarargument{OPT}{POLICY}{String constants to specify the memory
    allocation policy}
\end{envvararguments}

One, two, or more partitions can be specified using this environment variable.
Each partition is represented using a number called \emph{Partition ID}.
\textless ID\textgreater is the user-specified part of the environment variable
to represent the Partition ID. A maximum of SHMEM\_MAX\_PARTITIONS can be
defined and the Partition ID may be any non-zero positive integer between 1 and
SHMEM\_MAX\_PARTITION\_ID.

Each partition can take a maximum of four property specifiers as input. The
\emph{SIZE} specifier is the only required value. It is used to specify the
required symmetric heap size on the partition. The \emph{PGSIZE} specifier is
used to specify the page size of the memory kind.

On systems supporting multiple memory kinds, each memory kinds which are
identified by the implementation should be documented and made available as
string constants to the users. The \emph{KIND} specifier is used to specify the
memory kind to determine location of the memory partition.

The \emph{POLICY} is used to specify on how strictly the request optional 
specifiers \emph{PGSIZE} and \emph{KIND} are to be honored. The following are 
the accepted values for \emph{POLICY} specifier. 
\begin{itemize}
\item \emph{MANDATORY} Any unavailability of the user requested optional
specifier would result in program termination with an informative error during
symmetric heap creation on the defined partition.
\item \emph{PREFERRED} Any unavailability of the user requested optional 
specifier would allow the library to try to allocate the requested \emph{SIZE}
on the implementation defined location.
\item \emph{INTERLEAVED} Page allocations are interleaved across numa domains 
associated with the memory kind.
\item \emph{SYSDEFAULT} System defined memory policy is used.
\end{itemize}

\envvarnotes{
    SHMEM\_SYMMETRIC\_SIZE is supported with its current syntax and meaning.
    But, it is an error to use both SHMEM\_SYMMETRIC\_SIZE and any form of
    SHMEM\_SYMMETRIC\_PARTITION environment variables together.

    The total size of the symmetric heap is determined as the sum of sizes 
    of all \emph{SIZE} specifier in the defined memory partitions.

    All available page sizes and supported memory kinds be documented by the
    implementation along with the \emph{SYSDEFAULT} values for the optional
    properties.

    Partition ID 1 has a special usage. Symmetric data object allocations
    identified using the partition ID (with \textit{shmem\_kind\_malloc} and
    \textit{shmem\_kind\_align} routines) are allocated on the symmetric heap
    created over the partition with requested ID. Any generic allocations
    without the partition ID (using \textit{shmem\_malloc} and
    \textit{shmem\_align}) would result in using the symmetric heap created over
    partition with ID 1.}

\end{envvardefinition}

