All OpenSHMEM implementations supporting the use of multiple Symmetric Memory
Partitions provide support for a special environment variable called
SHMEM\_SYMMETRIC\_PARTITION. It defines the traits to establish the
characteristics of a memory partition. %These traits including the size of
%the symmetric heap are defined specifiers to the environment variable.

\begin{envvardefinition}
\begin{envvarname}
SHMEM_SYMMETRIC_PARTITION<ID>=SIZE=<size>[:PGSIZE=<pgsize>][:KIND=<kind>:POLICY=<policy>]
\end{envvarname}

\begin{envvararguments}
    \envvarargument{REQ}{ID}{Integer used to identify the partition}
    \envvarargument{REQ}{SIZE}{Number of bytes to allocate for symmetric heap}
    \envvarargument{OPT}{PGSIZE}{Number of bytes to specify the size of the
    page used by the partition}
    \envvarargument{OPT}{KIND}{String constants to specify the kind of memory
    if multiple memory kinds are supported by the system and identified by the
    implementation}
    \envvarargument{OPT}{POLICY}{String constants to specify the memory
    allocation policy used by the partition}
\end{envvararguments}

One, two, or more partitions can be specified using this environment variable.
Each partition is represented using a number called \emph{Partition ID}.
\textless ID\textgreater is the user-specified part of the environment variable
to represent the Partition ID. A maximum of SHMEM\_MAX\_PARTITIONS can be
defined. These defined partitions can have any non-zero positive integer
between 1 and SHMEM\_MAX\_PARTITION\_ID as Partition ID.

Each partition takes a maximum of four traits as input. The \emph{SIZE} is the
only required trait. It is used to specify the required symmetric heap size on
the partition. The \emph{PGSIZE} trait is used to specify the size of page used
by the partition.

The \emph{KIND} trait is used to specify the memory kind used by the partition.
On systems supporting multiple different kinds of memory, each memory kind
identified by the implementation should be documented and made available as
case-sensitive string constants to the users. Along with other identified
memory kinds, every implementation should define the following two kinds.
\begin{itemize}
\item \emph{NORMALMEM} The primary memory kind of a node
\item \emph{SYSDEFAULT} System defined memory kind is used
\end{itemize}

The \emph{POLICY} is used to specify on how strictly the user requested
optional traits (\emph{PGSIZE} and \emph{KIND}) are to be honored. The
following are the accepted values.
\begin{itemize}
\item \emph{MANDATORY} Any unavailability of the user requested optional
trait would result in program termination with an informative error message
during symmetric heap creation on the defined partition.
\item \emph{PREFERRED} Any unavailability of the user requested optional trait
would allow the library to try and allocate the requested \emph{SIZE} on an
implementation determined location.
\item \emph{INTERLEAVED} Page allocations are interleaved across NUMA domains
associated with the defined \emph{KIND}.
\item \emph{SYSDEFAULT} System defined default memory policy is used.
\end{itemize}

\envvarnotes{
    There are two types of memory allocations - generic (Refer
    \ref{subsec:shfree}) and partition-specific allocations
    (Refer \ref{subsec:shmem_kind}). Partition with ID 1 has a special usage.
    When multiple partitions are created, any generic allocations of symmetric
    data objects will use the symmetric heap created at partition with ID as 1.

    SHMEM\_SYMMETRIC\_SIZE is supported with its current syntax and meaning.
    But, it is an error to use both SHMEM\_SYMMETRIC\_SIZE and
    SHMEM\_SYMMETRIC\_PARTITION1 environment variables together.

    The total size of the symmetric heap is determined as the sum of all
    \emph{SIZE} traits in the defined memory partitions.

    All available page sizes and supported memory kinds should be documented
    by the implementation along with the \emph{SYSDEFAULT} values for the
    optional traits.
}


\end{envvardefinition}

